% !TEX root = ../CedricDe Schepper2023_Thesis.tex

\section{Results / Evaluation}\label{sec:results}

In order to evaluate the results from the experiment, we have to verify whether the phases perform their objective as described. Firstly, does the initialisation phase succeed in generating a `feasible´ solution namely one without or with a minimal number of hard constraint violations? Secondly, does the optimisation phase succeed in transforming the initial solution into one with an acceptable solution?

\subsection{Reference solutions}

The actual exam schedules, manually created by the administration of the \acrlong{ua}, can be used as a baseline to compare the generated schedules. These exam distributions can be seen in Figures \ref{fig:manual_sem1} and \ref{fig:manual_sem2}. Most notably, the manual schedules have a focus on keeping the amount of same day exams to a minimum and having 2 or more days in between exams as much as possible. This distribution is especially visible in the schedules for \acrshort{fti}. Additionally, it can be seen taht the schedule for \acrshort{fwet} contains a high amount of exam conflicts. In order to be considered a superior solution, automatically generated schedules will have to minimise the amount of exams with fewer than 2 days in between, with a focus on same day exam violations.

\begin{figure}[H]
  \centering
  \subfloat[FTI]{\includegraphics[width=0.5\textwidth]{images/manual/fti_sem1_percentages_manual.png}}
  \hfill
  \subfloat[FWET]{\includegraphics[width=0.5\textwidth]{images/manual/fwet_sem1_percentages_manual.png}}
  \caption{Manual exam distribution for January 2021}
  \label{fig:manual_sem1}
\end{figure}

\begin{figure}[H]
  \centering
  \subfloat[FTI]{\includegraphics[width=0.5\textwidth]{images/manual/fti_sem2_percentages_manual.png}}
  \hfill
  \subfloat[FWET]{\includegraphics[width=0.5\textwidth]{images/manual/fwet_sem2_percentages_manual.png}}
  \caption{Manual exam distribution for June 2021}
  \label{fig:manual_sem2}
\end{figure}

\subsection{Initialisation phase}

In order to quantify the feasibility of a solution, it is possible to look at either the objective function or the amount of hard constraint violations. Since the objective function was adapted to contain exponents and the conflict weight having a large impact on the size of the objective, looking at the amount of hard constraint violations can provide a better perspective. By plotting the violations per iterations, the progress made by the initialisation phase can be visualised. Figure \ref{fig:violations} shows that the initialisation phase is able to generate a solution for \acrshort{fti}+\acrshort{fwet} without hard constraint violations. It reaches a feasible solution after fewer than 150 iterations, generally taking under 500 seconds. 

However, Figure \ref{fig:init} shows that no attention was given to the distribution which can be confirmed by the presence of a high percentage of exams after a single day. These figures are generated by calculating the distance between each exam for every student, with the exams being sorted on exam date. For example, a student A with exams on day 1, 5, and 7 will contribute the distance between day 1 and 5, as well as the distance between day 5 and 7. This results in the student contributing to the $x = 4$ and $x = 2$ bar, respectively. We can also look at the case for a student B with conflicting exams. This can be showcased using an identical schedule compared to student A, with the addition of a conflicting exam on day 5. This results in the distances being calculated using the sequence of 1, 5, 5, and 7. Because of this, student B will contribute to $x = 4$, $x = 0$, and $x = 2$ bar, respectively.

\begin{figure}[H]
	\centering
	\includegraphics[width=0.5\textwidth]{images/init/conflicts.png} 
	\caption{Hard constraint violations}
	\label{fig:violations}
\end{figure}
% TODO add dotted line for y=0
\begin{figure}[H]
  \centering
  \subfloat[FTI]{\includegraphics[width=0.5\textwidth]{images/init/fti.png}}
  \hfill
  \subfloat[FWET]{\includegraphics[width=0.5\textwidth]{images/init/fwet.png}}
  \caption{Exam distribution after initialisation phase for June 2021}
  \label{fig:init}
\end{figure}

\subsection{Optimisation phase}

While we have shown that tabu search can efficiently generate a feasible solution from the provided data set, the exam distribution will determine whether the automated schedules are superior in quality compared to the 
% Table~\ref{tab:example} shows an example of creating a table.

% \begin{table}[h]
% 	\caption{Fictitious Results}
% 	\label{tab:example}
% 	\centering
% 	\begin{tabular}{l c c c }
% 		\hline
% 		& \textbf{Precision} & \textbf{Recall} & \textbf{F-Score} \\ \hline
% 		Technique 1 & 0.85 & 0.77 & 0.81 \\
% 		Technique 2 & 0.82 & 0.81 & 0.81 \\
% 	         Technique 3 & 0.65 & 0.93 & 0.73 \\ \hline
% 	\end{tabular}
% \end{table}


