\documentclass{article}
\usepackage{arxiv}
\usepackage[utf8]{inputenc}
\usepackage[T1]{fontenc}    % use 8-bit T1 fonts
\usepackage{hyperref}       % hyperlinks
 \usepackage{backref} % back references from the bibliography
\usepackage{url}            % simple URL typesetting
\usepackage{booktabs}       % professional-quality tables
\usepackage{amsfonts}       % blackboard math symbols
\usepackage{amsmath}
\usepackage{nicefrac}       % compact symbols for 1/2, etc.
\usepackage{microtype}      % microtypography
\usepackage{lipsum}
\usepackage{cite} % organizefs citations
\usepackage{graphicx}
\usepackage{float}
\usepackage{subcaption}
\usepackage[ruled, linesnumbered]{algorithm2e}
\usepackage{cleveref}
\usepackage{longtable}
\usepackage{algorithm2e}
\usepackage[acronym]{glossaries}
\usepackage{enumitem}

\usepackage{ textcomp }
\usepackage{scalerel,amssymb}
\def\mcirc{\mathbin{\scalerel*{\bigcirc}{t}}}
\def\msquare{\mathord{\scalerel*{\Box}{gX}}}


\makeglossaries

%\usepackage[disable]{todonotes} %% Uncomment this line and comment the next one to make the notes invisible
\usepackage{todonotes}
\newcommand{\comment}[1]{\todo[color=orange!40, inline]{\footnotesize{#1}}}
\newcommand{\student}[1]{\todo[color=green!40, inline]{\footnotesize{Student: #1}}}
\newcommand{\serge}[1]{\todo[color=purple!40, inline]{\footnotesize{Serge: #1}}}



\hypersetup{
    colorlinks,%
    citecolor=black,%t
    filecolor=black,%
    linkcolor=black,%
    urlcolor=black,
%
%    pdftitle={XXX},    % title
%    pdfauthor={XXX},     % author
%    pdfsubject={XXX},   % subject of the document
%    pdfcreator={PDF LaTex},   % creator of the document
}
\backrefsetup{verbose=false,hyperpageref}


\title{Tabu Search for optimising the timetables of university exams}
\author{De Schepper Cedric}
\supervisor{Principal Advisor: Prof. Dr. Serge Demeyer}
\assistant{Assistant Advisor: Joey De Pauw}
\date{\today}


\begin{document}
% TODO change to submission month and year

\maketitle

\newpage
\tableofcontents
\newpage
\listoffigures
\newpage
\listoftables

\newpage
% !TEX root = ../CedricDe Schepper2023_Thesis.tex

\section*{Abstract}
\addcontentsline{toc}{section}{Abstract}

The exam timetabling problem faced by the \acrfull{ua} is a complex task requiring the scheduling of exams within constraints. The current process is manual and resource-intensive, resulting in lost time and imperfect exam timetables. We propose a \acrlong{tabu} implementation efficient in generating feasible exam timetables. This significantly reduces the amount of time required to generate qualitative time tables.

\newpage
% !TEX root = ../CedricDe Schepper2023_Thesis.tex

\section*{Nederlandstalige Samenvatting}
\addcontentsline{toc}{section}{Nederlandstalige Samenvatting}

Het creëren van een examenrooster door de Universiteit Antwerpen is een complexe taak waarbij alle examens ingepland moeten worden, rekening houdend met bepaalde vereisten. Het huidige proces is handmatig en arbeidsintensief, wat resulteert in verloren tijd en imperfecte examenroosters.
Wij stellen een Tabu Search implementatie voor die efficiënt is in het opstellen van aanvaardbare examenroosters. Dit vermindert aanzienlijk de manuele tijd die nodig is om kwalitatieve examenroosters te bekomen.


\newpage
% !TEX root = ../CedricDe Schepper2023_Thesis.tex

\section*{Acknowledgements}
\addcontentsline{toc}{section}{Acknowledgements}

I would like to express my gratitude to several people for supporting me throughout the writing of this master's thesis. First, I would like to thank Prof. Dr. Serge Demeyer for giving me the opportunity to perform research in this interesting and relevant subject. A special thanks goes to Joey De Pauw, whose continuous feedback and support was crucial to finishing my thesis. His support ranged from giving invaluable suggestions, to helping me stay on track, to proofreading every intermediate version. 

Additionally, I want to thank Annelies Barentsen, Heidi Snellings, and Annick Van Son from the administration of the \acrlong{ua}. During several meetings, they shared their expert knowledge into the difficulties and requirements they faced when creating the necessary exam schedules. Their insights were essential to implementing my timetabling algorithm and interpreting the results.

Furthermore, I couldn't have done this without the support of Bram De Schouwer and Niels Boone. Completing this thesis combined with working full-time at Deloitte has not been easy on me. Because of this, I'm very grateful for their support, especially for giving me the flexibility to take some days off when needed so I could focus fully on my research. 

Finally, I'd like to thank my family and friends for their never-ending support. Especially my parents and my sister Aline for supporting me during my entire academical journey. Also Jolien, for helping me stay motivated and always being there to encourage me.


\newpage
% !TEX root = ../CedricDe Schepper2023_Thesis.tex

\section{Introduction}\label{sec:introduction}

% The introduction is probably the most important section of any academic work. 
% We start the introduction with a small contextualization on the thesis/paper subject. 
% This usually takes 1 to 3 paragraphs for a paper depending on the topic. 
% For a thesis, it is ok to write more paragraphs. 
% Please remember, these are just general suggestions.

% After the contextualization, we write one paragraph on the problem or motivation for this research. 
% We can complement with another paragraph to reinforce why the problem is important, or how it affects academia and/or industry. 

% Citations are very important in academic writing. 
% Try to put at least one citation (preferable more) per paragraph in the introduction's previous paragraphs. 
% Always use a citation when making a strong remark or statement to reinforce the point.
% Example of citation~\cite{demeyer2002}. For multiple citations put them all in the same cite command~\cite{vanbladel2020, parsai2020, njima2019, demeyer2002}. 
% Remember that citations are annotations, not parts of speech.
% Therefore do not use a citation as a substantive.

% After we successfully introduced the readers to the contextualization and problem/motivation, comes a paragraph clearly stating what is our research. 
% Usually, this paragraph begins with "In this paper/thesis, we ...".

% Now the reader understands the basics of our research and what we did to accomplish our goals. 
% The remaining paragraphs in the introduction can now describe a summary of the results, how previous research does not tackle what we did/accomplish, state the contributions for the research, or even an illustrative example of how the research improves the problem we described. 

% For Git-like repositories, try to put each sentence in a newline. 
% Since Git is line-based, it makes it easier the see changes between versions.

% The final paragraph of the introduction is an outline briefly describing the remaining sections. 
% Use the \textbackslash ref\{...\} command to reference Sections. 
% For example, in Section~\ref{sec:background}, we describe...



University exam timetabling is an unsolved problem encountered by every university's administration \cite{even1976}. Every year, significant manual effort has to be put into the creation of exam timetables. This need for automated tools capable of generating acceptable solutions has attracted the attention of researchers since 1963 \cite{gotlieb1963}. 

This task of creating exam timetables can be translated into a scheduling problem \cite{BurkeScheduling2004}. The goal is to assign all exams to available time slots and suitable rooms in order to produce a schedule without conflicts. Additionally, the distribution of exams has to be optimised as to provide a student with the highest chance of passing an exam. These requirements are generally defined as hard and soft constraints. Hard constraints are requirements that have to be met in order to be considered a feasible solution, while soft constraints are preferred to be violated as little as possible. Not all timetabling problems might have feasible solutions depending on the data set used.

The current relevance of this problem is showcased by the amount of algorithms developed, and the organisation of both timetabling conferences and competitions. The survey by Carter \cite{carter1986} details the research performed before 1986, while surveys such as the one by Siang Tan et al. \cite{joo2021} describe more recent implementations. However, no known research has been performed on the timetabling problem using the data and constraints of the University of Antwerp.

In this paper, we add a first contribution to the specific timetabling problem of the University of Antwerp. We do this by translating the exact use case into a formal scheduling problem. In order to solve the scheduling problem, we investigate which algorithm holds the most potential for this specific problem. This is done by reviewing the different types of algorithms available \cite{joo2021, kristiansenSurvey2013, chen2021, rong2009}. After proposing the implementation of a customised Tabu Search algorithm, we analyse the performance of this implementation. This is done by answering two questions based on quantitative and qualitative data. First, does Tabu Search succeed in creating a feasible solution without hard constraint violations? Second, can we optimise the solution to provide an acceptable exam distribution?

In Section \ref{sec:problem}, we look at the timetabling problem in detail, describing the constraints applied, its complexity, and the data required. Next, we look at the research already performed on this topic in Section \ref{sec:related-work}, including the benchmarks available and the different types of search algorithms applied. Section \ref{sec:method} focuses on the Tabu Search algorithm chosen and its implementation. Section \ref{sec:experiment} details the scoring metrics used to compare solutions and looks at fine tuning the needed hyper parameters. Finally, sections \ref{sec:results} and \ref{sec:conclusion} analyse the results obtained by executing several analyses, resulting in our final conclusions. 

\newpage
% !TEX root = ../CedricDe Schepper2023_Thesis.tex

\section{Problem definition}\label{sec:problem}

 \subsection{Problem definition}

The task of creating a university exam timetable can be reduced to a scheduling problem. The goal is to assign all exams to available time slots in order to produce a schedule without conflicts. Additionally, the distribution of exams has to be optimised as to provide a student with the highest chance of success. These requirements generally are defined as hard and soft constraint. Hard constraints are requirements that have to be met in order to considered a feasible solution, while soft constraints are preferred to be violated as little as possible. Not all timetabling problems might have feasible solutions depending on the data set used. After consulting with the responsible parties for the scheduling of the exams for the FTI and FWET faculties, the following constraints were determined:
\begin{itemize}
    \item Constraint 1: Each exam is scheduled to a timeslot.
    \item Constraint 2: Each exam is scheduled to a room of the correct type.
    \item Constraint 3: The number of students enrolled in an exam cannot exceed the capacity of the scheduled room.
    \item Constraint 4: No student must be scheduled to more than 1 exam on the same day.
    \item Constraint 5: The number of days between exams must be maximised
\end{itemize}
Constraint 1 to 4 are considered to be hard constraints with contraint 5 being the sole soft constraint. Formally this means that the first 4 constraints must be satisfied in order to provide a feasible solution, with constraint 5 being taken into account to improve it.

In order to solve this scheduling problem, some data must be provided or generated:

\begin{description}
   \item [Students] The set of all students that are enrolled in exams and will impact the timetabling.
   \item[Exams] The set of all exams that must be scheduled. Every exam has its own set of students that are enrolled which will allow to verify if two exams have common students and would thus conflict if they occur on the same day. Additionally, an exam belongs to a faculty and is of a certain type. Two examples of exam types are oral and written exams.
   \item[Rooms] The set of rooms present for examinations. Each room has a room type, a student capacity and  list of faculties it is available for. An exam can only be scheduled in a room if the capacity is sufficient, the type of exam is possible within the room, and the room is available to the faculty to which the exam belongs.
    \item[Time slots] The set of all time slots that can be used to schedule exams. A time slot is determined by its date, exam time, and room. The exam time determines the amount of times a room can be used per day.
    \item[Periods] Set of periods, one for each date within the examination schedule. Each period contains all time slots of that date. By checking all students that are enrolled within exams scheduled to a time slot of a period, the amount of constraint 4 ("no student must be scheduled to more than 1 exam on the same day") violations can be determined. These periods can be generated by adding a time slot for every room and time combination. The amount of time slots in a period corresponds to $|\text{Rooms}| * |\text{Exam Times}|$.
\end{description}



\newpage
% !TEX root = ../CedricDe Schepper2023_Thesis.tex

\section{Related Work}\label{sec:related-work}

This section will detail different types of algorithms researched to solve the timetabling problem.

\begin{description}
    \item [Integer programming] Integer programming tries to solve an optimisation by minimising or maximising the problem's objective function. Fundamentally, some or all required variables are constrained to integers. Since integer programming makes use of an exhaustive search process, the search space must be reduced as much as possible to reduce the run time.
    \item [Simulated Annealing] Simulated Annealing\cite{kirkpatrick1983} makes use of a temperature variable which describes the level of randomness present in the acceptance of a new solution. After generating an initial solution, newly obtained solutions are accepted based on the change in objective function and current temperature. Over time the temperature cools down which reduces the amount of randomness.
    \item [Adaptive Large Neighbourhood Search] Adaptive Large Neighbourhood Search (ALNS) \cite{ropke2006} works by generating new solutions by constantly making changes to the current solution in order to explore the neighbourhood space. These changes are obtained by removing part of the variables and then reintroducing new values in order to generate neighbours. ALNS is considered adaptive since it keeps track of the performance of certain operations and adjusts its parameters accordingly. The use of ALNS for exam timetabling so far has been limited.
    \item [Genetic Algorithms] Genetic Algorithms (GA) \cite{Holland1975} are based on the process of natural selection witnessed in nature. They work by generating an initial set or population of possible solutions. Iteratively, a new population will be formed by selecting the fittest (best) solutions and combining two solutions to create a new generation of solutions.
    \item [Particle Swarm Optimisation] Particle Swarm Optimisation (PSO) \cite{kennedy1995} is derived from the behaviour by collective species as seen in fish schools and bird flocks. A swarm of particles (see fish or bird), representing solutions, moves through the search space at a certain speed and direction.
    \item [Honey-Bee Mating Optimisation] Honey-Bee Mating Optimisation algorithms (HBMO) \cite{abbas2001} are based on the mating of honey-bees as witnessed in nature. During the mating procedure, the queen representing the current best solution is able to mate with other bees in order to produce new solutions.


\end{description}

\subsection{Integer programming}

Kristiansen et al. \cite{kristiansen2015} describes a mixed-integer linear programming (MIP) model designed to solve XHSTT timetabling data sets. The XHSTT format was formulated to standardise timetabling data sets in order to compare heuristics. The proposed model makes use of two stages. During the first stage, a simplified MIP model is generated taking only the hard constraints of the problem into account. This stage is ran until a specified amount of time has passed or until the model has been solved to optimality. The benefit of using integer programming here is that MIP models can issue 'certificates of optimality, indicating that an optimal solution has been reached. This differs from heuristic methods where a model cannot guarantee that an optimal solution has been found unless the objective function is brought to 0. With MIP models, one can determine whether the cost generated by the hard constraints is the most optimal solution feasible. This allows a clear cut off point for the model to stop focusing on the hard constraints solely. 

If a certificate of optimality can be generated, the second stage is executed. Otherwise, the algorithm ends. Before solving the model in stage 2, the soft constraints are added again. Additionally, an extra constraint is added which keeps the optimal value of the cost generated by the hard constraints. The second stage ends after the remaining time after stage 1 has passed. The proposed model was not only successful in creating 2 new solutions to XHSTT data sets, it was also able to prove optimality of previously found solutions. Additionally, it could also provide the lower bounds for several other data sets and improve the best solution found so far. These lower bounds are crucial in order to compare the quality of solutions that have not reached optimality.

Al-hawari et al. \cite{hawari2017} attempts to solve the university exam timetabling problem by splitting the problem into three smaller sub-problems in order to significantly decrease the amount of variables required in the formulas used and thus reduce the processing power and storage capacity needed. Additionally, the simplification of the formulas also improves the explainability of the model, making it easier to understand. The initial problem is split into 3 problems, each sub-problem continuing on the solution provided by the previous phase. Phase one and two will use a  graph colouring IP model to generate a feasible solution. Graph colouring works by assigning colours to the vertices while avoiding that two vertices connected by an edge are assigned the same colour. These vertices are called adjacent. Additionally, graph colouring will attempt to use the least amount of colours possible to generate a feasible solutions.

The first phase starts by assigning time slots to the different exams. Here, exams are seen as the vertices with two vertices being connected by an edge if a student is enrolled in both exams. The time slots act as colours and are assigned to the vertices. Every feasible solution produced by this phase will satisfy the hard constraint that a student can not have more than one exam at the same time. Phase two builds upon this solution by assigning days to the time slots, meaning that the time slots are now considered vertices and the different days colours. Again, vertices are adjacent if the exams assigned to the time slots share mutual students. The new solution will have added the extra constraint that a student can't have two exams on the same day. Lastly, the third stage will assign all exams to rooms, keeping into account the exam enrolment and room capacity. The final solution, if feasible, will satisfy all hard constraints. An important remark to note is that no soft constraints were introduced for creating an improved exam distribution. As a result, a feasible solution will only make sure that a student has no exams on two following days or on the same day.


\subsection{Simulated Annealing}

Aycan and Ayav \cite{aycan2009} apply Simulated Annealing in order to generate optimal solutions. In order to create a initial solution as optimal as possible, constraint satisfaction methods are used to create a solution satisfying all hard constraints. During the simulated annealing phase, a new solution is found by randomly swapping two variables. The cost of the new solution is then calculated using the objective function. Whenever the new objective is lower than the objective of the of the previous solution, the algorithm will keep the new solution. In the case of a higher objective, the temperature will determine based on the difference between the two costs whether to keep the new solution or discard it. The possibility of accepting worsening solutions allows the algorithm to avoid being stuck in local minima. Over time the temperature will reduce based on the cooling schedule. Practically, the allowed difference in cost for worsening solutions in order to still be accepted will decrease. As the temperature decreases, the focus switches from exploring to exploiting the search space. This will allow the algorithm to eventually converge towards a local or global minimum. The performance of Simulated Annealing is determined by the choice of initial solution, the objective function, and cooling schedule. 

The objective function accounts for the impact of both hard as well as soft constraints. Every constraint is assigned its own penalty function including a constraint weight. The cooling schedule proposed makes use of geometric schedule. That means that the temperature will decrease by a constant factor during every step of the algorithm. The choice for the initial temperature and cool-down factor will determine the share of the search space visited. 

While geometric cooling is a very simple to implement, it comes with some shortcoming. Since the temperature is calculated by a deterministic schedule, it mostly depends on the variables chosen by the user. Additionally, this schedule also does not take into account the progress made by the algorithm. Alternative, the more complex adaptive cooling schedules will decrease, or even increase, the temperature based on the rate of acceptance for new solutions. While Aycan and Ayav conclude that this method succeeds in satisfying the hard constraint in order to find a feasible solution, implementing a hybrid version or adding reheating to the cooling schedule might obtain higher quality results.

A more complex cooling schedule can be seen in the Simulated Annealing with Reheating (SAR) algorithm by Leng Goh et al.\cite{Goh2017} They propose a two stage hybrid timetabling algorithm where the first stage uses Tabu search to generate a feasible solution. If such a solution is found, the second stage attempts to improve on it by using SAR. Instead of using geometric cooling, reheating or increasing the temperature is possible. This is based on the assumption that whenever the objective is high, the focus should be on exploration and accordingly whenever the objective is low, exploitation is prioritised. Whenever the search is estimated to be stuck in a local optima, the temperature will be reheated until it manages to escape. 

This estimate is made by checking the difference between the best and current objective. If the change in objective is under a certain threshold for a pre-determined amount of iterations, the search is considered to be stuck and reheating will occur. This cooling and reheating repeats itself until an optimal solution has been found or a step limit is reached. An additional benefit to reheating compared to a geometric schedule is that no fine-tuning of variables is needed when extending the runtime. Figure \ref{fig:SAR} showcases the effect that enabling reheating has on the temperature, with its value increasing whenever it gets stuck in a local optimum. As a consequence, the search is allowed to explore more, resulting in a higher amount of variation of the objective. In this case, the reheating allowed the search to discover a more optimal solution.

\begin{figure}[h]
  \centering
  \subfloat[SAR with reheating disabled]{\includegraphics[width=0.5\textwidth]{images/related_works/SA/SAR_disabled.png}}
  \hfill
  \subfloat[SAR with reheating enabled]{\includegraphics[width=0.5\textwidth]{images/related_works/SA/SAR_enabled.png}}
  \caption{Effect of reheating on the temperate and objective}
  \label{fig:SAR}
\end{figure}

\subsection{Adaptive Large Neighbourhood Search}

S{\o}rensen and Stidsen \cite{sorensen2012} propose a version of ALNS building on the more general Large Neighbourhood Search algorithm. LNS works by creating new solutions by applying a "destroy" and "repair" operation. Every step a destroy operation will remove a set of variables from the problem, before reintroducing new variables in order to create new solutions. By changing multiple variables, LNS is able to escape local minima. In the proposed ALNS extension however, the single destroy and repair operations are replaced by multiple operations, chosen at random during execution. This changes the original deterministic model into a stochastic version, introducing randomness. Additionally an adaptive layer analyses the impact of each operator and increases the probability of operators having a positive impact on the objective function.

\subsection{Genetic Algorithms}

Genetic Algorithms (GA) are based on the process of natural selection witnessed in nature. It generally consists of several base steps before reaching a final solution. Firstly, the algorithm creates an initial population consisting of feasible solutions. An iterative approach is then taken in order to evolve this population into new generations. For every generation, the fitness of each individual in the population is determined and the 'fittest' individuals are chosen as parents for the new generation. In order to obtain new offspring, genetic operators such as crossover and mutation are applied. Crossover works by combining information of parent solutions into a new offspring. This can be done by swapping exam assignments across the parents. Mutation on the other hand randomly changes assignments in order to create new offspring. This is done in order to maintain randomness, allowing GA to escape local minima. The iterative flow of genetic algorithms can be seen in figure \ref{fig:GA}.

\begin{figure}[h]
	\centering
	\includegraphics[width=0.35\textwidth]{images/related_works/GA/GA.png} 
	\caption{Iterative approach by genetic algorithms}
	\label{fig:GA}
\end{figure}

% TODO: add GA drawing (GA.drawio)

The paper submitted by Pillay and Banzhaf \cite{pillay2010} showcases a two-phased approach to create feasible timetables. Both phases utilise a genetic algorithm in order to come to a solution, with the first phase focusing on producing a timetable that does not violate any hard constraints. The second phase later attempts the optimise the objective of the soft constraints. Up until 2007, most papers applying genetic algorithms used data sets specific to a particular institution. This algorithm is tested on the Carter benchmarks, allowing its performance to be compared to different approaches.

During the first phase, domain specific heuristics are used instead of random operations in order to create the initial population. While these heuristics requires domain knowledge, the quality of the initial population should be superior. In order to determine the next exam to be assigned, the obtained heuristics make use of the number of conflicts, the number of students enrolled per exam, or the number of students with conflicts. These heuristics were then compared to random and best-slot scheduling. It was found that using heuristics succeeded in lowering the soft constraints objective compared to other scheduling methods. For the iterative steps, the fitness function is described as the amount of conflicts per timetable. In order to select the parents with which to generate the new generation, tournament selection is applied. Here, a determined amount of individuals are randomly chosen from the population to be compared against each other. The individual having the lowest objective ends up being selected. This process is repeated until until all parents have been selected. When creating new offspring, both mutation and crossover operators were considered. During mutation, a random amount of conflicting examinations are rescheduled. Moreover, crossover operations that randomly swap time slots between two parents were tested. This requires an additional repair mechanism to remove duplicate examinations and schedule missing examinations. Since the crossover operations did not have a positive impact on the performance, only mutation was used.

The second phase focuses on minimising the objective of the soft constraints in order the generate a more optimal timetable. While very similar to the first phase,  mutation operations are now continuously applied until an offspring superior to the parent is found or until a certain amount of iterations has been reached. The performance of this genetic algorithm was eventually compared to other studies using the same benchmark including Tabu search and large neighbourhood search. While no optimal solutions were found for any of the data sets, equal or improved solutions were obtained compared to alternative methods.

Chu and Fang \cite{Chu2000} made a comparison between using genetic algorithms versus Tabu search to obtain time tables. For all of their different experiments, the Tabu search implementation was able to outperform their genetic algorithm both on the quality of the solution as the computational time needed to converge. However, a redeeming quality of Tabu search is that it is able to produce several near optimal solutions in one go while Tabu search is limited to a single solution.
\subsection{Particle Swarm Optimisation}

Chen and Shih \cite{Chen2013} discuss a particle swarm optimisation implementation combined with local search %TODO expand on this
During PSO, a swarm of particles, corresponding to different timetables, moves through the available search space. During every iteration, each particle will remember its best encountered position and share its information with the rest of the swarm. The particles will then adjust their velocity and direction on both their personal and global best. The velocity can be seen as the magnitude of the allowed change per time step. In order to avoid particles being trapped in a local optima, an iteration of local search is executed after a particle move. This step will check the surrounding areas for a better solution. 

Tassopoulos and Beligiannis \cite{Tassopoulos2012} builds on the general Particle swarm optimisation algorithm. Initially, they start with a high amount of 150 particles. Whenever the fitness value of a produced particle exceeds a certain tolerance value, this particle is set to inactive and will not be used in further steps. This tolerance value is calculated during each step in order to deactivate weak particles at the start but make it harder for particles to reach that threshold down the line. When the amount of particles is reduced to 30, this procedure stops. The reasoning behind starting with a large amount of particles is to make it possible to explore a wider search space while keeping the execution time down by reducing the particles over time. Similar to Chen and Shih's approach \cite{Chen2013} a local search algorithm is used to minimise one of the soft constraints. Results show that it outperforms the genetic and evolutionary algorithms that the particle swarm optimisation implementation was tested against.
\subsection{Honey-Bee Mating Optimisation}

Sabar et al. \cite{Sabar2009} are the first to propose the use of Honey-bee mating optimisation in order to solve the exam timetabling problem. Since specific terms used in the natural mating process are used when describing the optimisation algorithm, table \ref{tab:hbmo} provides translation of the analogy. The original HBMO algorithm starts follows the natural process when the queen leaves the hive on a mating flight. Here she maintains a certain speed and direction, creating the possibility of drones to mate with her. After mating with a drone, its genetic information is stored within the queen to be used in the breeding phase. After each mating, the queen will change her energy and speed. As soon as the queen reaches a certain energy threshold or reaches a mating limit, she will return to the hive. Upon return, the queen will randomly select genetic information obtained and perform a crossover step to create a new brood. Every brood is fed by a worker in order to improve the obtained solution. After every brood is improved, the fittest brood is compared to the queen. If the brood corresponds to a superior solution, the queen is replaced by the brood. Both the queen and the other broods are killed and a new mating flight will commence. 

The variant on the original HBMO algorithm proposed by Sabar et al. attempts to solve HBMO's weakness of suffering from early convergence. In the original algorithm, the drones used to mate with the queen are never replaced which reduces the amount of variation. This is solved by replacing the drones that were successful in mating with the newly created broods. Additionally, after crossover, the heuristic applied by the worker is based on local search in order to optimise the brood as much as possible.

While HBMO is very similar to other population based methods such as genetic algorithms, two clear differences can be noted. HBMO maintains the queen as one of the two 'parents' used during crossover. Since the queen is considered as the current best solution, this is meant to improve obtained solutions. Secondly, the local search applied by the worker can be considered an exploitation phase which is not present in genetic algorithms. Finally, they conclude that the proposed HBMO alternative manages to create comparable or superior solutions compared against other population based methods.

\begin{table}[h]
	\caption{Analogy between the natural mating process and the HBMO algorithm}
	\label{tab:hbmo}
	\centering
	\begin{tabular}{l c c }
		\hline
		\textbf{Natural honey bee}  & \textbf{Artificial honey bee} \\ \hline
		Queen & Current best solution \\
		Drones & Possible solutions \\
	    Broods & Newly generated solutions \\
            Worker & Heuristic search \\
            Mating or Breeding & Crossover \\ \hline
	\end{tabular}
\end{table}




%\newpage
%% !TEX root = ../CedricDe Schepper2023_Thesis.tex

\section{Background}\label{sec:background}

% In a Background section, we describe the main concepts and/or techniques that are important for a reader to better understand the experiments. Usually, we divide the background into several subsections, one for each concept/technique.

% Figure~\ref{fig:ua-logo} is just an example of how to use the figure command.
% \begin{figure}[h]
% 	\centering
% 	\includegraphics[width=0.3\textwidth]{images/ua.jpg} 
% 	\caption{University of Antwerp - Logo}
% 	\label{fig:ua-logo}
% \end{figure}

\subsection{Tabu Search}

Tabu Search is a metaheuristic search algorithm based on local search, a heuristic optimisation method that traverses the possible search space by performing local changes to the current solution. Since local search will only accept improving solutions, this traversal can end up being stuck in a local optimum. Tabu search is different from local search by relaxing this rule and also accepting worsening solutions. Additionally, tabu search maintains a memory structure to avoid changes being reversed. The usage of short- to long-term memory is  based on the assumption that optimisation techniques must incorporate memory to qualify as intelligent and that a bad strategic choice is superior to a good random choice\cite{glover1999}. This memory structure is implemented by maintaining a tabu list which contains the x most recent changes performed. A change is thus considered 'tabu' if it is present within the tabu list.
\\\\
Tabu search starts by constructing an initial solution. The solution can be generated randomly or by applying a deterministic approach. During the entire process, the best seen solution to date is maintained. This is necessary due to tabu search allowing worsening changes or so called moves to avoid getting trapped in local minima. After the solution initialisation, the iterative procedure starts searching for a feasible solution. This loop ends when  a specified stopping condition is met. This condition is generally a combination of a solution its objective function scoring below a certain threshold or the procedure reaching a maximum number of iterations.
\\\\
The first step of the main loop is to generate the complete list of possible neighbours of the current solution and ranking them based on the objective function. Subsequently, the best neighbour that is either not tabu or that meets the aspiration criterion is chosen as the next solution. A neighbour. A neighbour is considered tabu if it is present within the tabu list. The aspiration criterion is added to be able to override the tabu requirement. A possible aspiration is to accept solutions that are better than the best seen solution. After choosing the next solution, the best seen solution is updated if the newly generated solution is superior based on the objective function.
\\\\
When the stopping criteria is met, the algorithm ends and the best solution is returned. The full flow of the algorithm can be seen in Fig.\ref{fig:tabu-chart}.

\begin{figure}[h]
	\centering
	\includegraphics[width=0.5\textwidth]{images/tabu.drawio.png} 
	\caption{Tabu search flow}
	\label{fig:tabu-chart}
\end{figure}


 




\newpage
% !TEX root = ../CedricDe Schepper2023_Thesis.tex

\section{Method / Experimental Design}\label{sec:method}

 \subsection{Problem definition}

The task of creating a university exam timetable can be reduced to a scheduling problem. The goal is to assign all exams to available time slots in order to produce a schedule without conflicts. Additionally, the distribution of exams has to be optimised as to provide a student with the highest chance of success. These requirements generally are defined as hard and soft constraint. Hard constraints are requirements that have to be met in order to considered a feasible solution, while soft constraints are preferred to be violated as little as possible. Not all timetabling problems might have feasible solutions depending on the data set used. After consulting with the responsible parties for the scheduling of the exams for the FTI and FWET faculties, the following constraints were determined:
\begin{itemize}
    \item Constraint 1: Each exam is scheduled to a timeslot.
    \item Constraint 2: Each exam is scheduled to a room of the correct type.
    \item Constraint 3: The number of students enrolled in an exam cannot exceed the capacity of the scheduled room.
    \item Constraint 4: No student must be scheduled to more than 1 exam on the same day.
    \item Constraint 5: The number of days between exams must be maximised
\end{itemize}
Constraint 1 to 4 are considered to be hard constraints with contraint 5 being the sole soft constraint. Formally this means that the first 4 constraints must be satisfied in order to provide a feasible solution, with constraint 5 being taken into account to improve it.

In order to solve this scheduling problem, some data must be provided or generated:

\begin{description}
   \item [Students] The set of all students that are enrolled in exams and will impact the timetabling.
   \item[Exams] The set of all exams that must be scheduled. Every exam has its own set of students that are enrolled which will allow to verify if two exams have common students and would thus conflict if they occur on the same day. Additionally, an exam belongs to a faculty and is of a certain type. Two examples of exam types are oral and written exams.
   \item[Rooms] The set of rooms present for examinations. Each room has a room type, a student capacity and  list of faculties it is available for. An exam can only be scheduled in a room if the capacity is sufficient, the type of exam is possible within the room, and the room is available to the faculty to which the exam belongs.
    \item[Time slots] The set of all time slots that can be used to schedule exams. A time slot is determined by its date, exam time, and room. The exam time determines the amount of times a room can be used per day.
    \item[Periods] Set of periods, one for each date within the examination schedule. Each period contains all time slots of that date. By checking all students that are enrolled within exams scheduled to a time slot of a period, the amount of constraint 4 ("no student must be scheduled to more than 1 exam on the same day") violations can be determined. These periods can be generated by adding a time slot for every room and time combination. The amount of time slots in a period corresponds to $|\text{Rooms}| * |\text{Exam Times}|$.
\end{description}


\subsection{Implementation frameworks}

The algorithm implemented to run the experiments is written using Python3.10. The code requires no packages that are not in the Python Standard Library. The profiler cProfiler is used to analyse the performance. It generates a list detailing how long each part of the algorithm runs for and allows to pinpoint and improve bottlenecks.

\subsection{Implementation: version 1}
For this thesis, the initial version or version 1 of the tabu search algorithm to be used in the experiments is based on the algorithm proposed by Alvarez-Valdes et al \cite{alvarez1997}. The algorithm described makes use of a 2-phased approach. During phase 1, the emphasis is on the hard constraints. More specifically, reducing the amount of common students conflicts on the same day is prioritised. Optimally, this initialisation phase would return a solution that is already feasible. The second phase can be considered an optimisation phase, which tries to satisfy the soft constraints as much as possible and thus attempt to distribute the exams evenly.

\subsubsection{Initialisation phase}
The algorithm starts with the initialisation for which the pseudo code is shown in algorithm \ref{alg:phase1}. Firstly, the initial solution must be generated. In the proposed algorithm this is done by randomly assigning all exams to time slots. Generally this will result in several hard constraint violations such as students have more than 1 exam on the same day and room capacity being exceeded. Since Alvarez-Valdes et al consider the room capacity a soft constraint, this is acceptable and will be improved during the iterations. However, in this case, room capacity is seen as a hard constraint that can't be violated at all. In order to circumvent these capacity violations, exams are sorted by student count and only then randomly assigned to time slots with rooms having sufficient capacity. As long as the time slot quantity and room capacity is sufficient, this will result in a solution with no room capacity violations. Another added complexity is the presence of different room types which was not the case for Alvarez-Valdes et al. In order to satisfy constraint 2, only rooms with a suitable type are considered when assigning exams to time slots.

\begin{algorithm}
 Generate an initial solution s in the space of solutions X\;
 $s^* = s$ (with $s^*$ the best solution seen so far)\;
 $k = 0$ (with k the number of iterations)\;
 \While{k < maximum number of iterations and $f(s^*) \neq 0$}{
  $k = k + 1$\;
  \tcc{perform single moves of one exam to another time slot}
  Generate set of solutions $V^* \subseteq N(s, k)$\ (set of neighbours of s)\;
  Sort $V^*$ by ascending $f(s')$ and select the best $s'$\;
   \ForEach{solution $s'$ in $V^*$}{
    \If{not $tabu(s')$ or $f(s') < f(s^*)$ (aspiration criterion)}{
        $s = s'$\;
        \textbf{break}\;
    }
   }
   \If{$f(s') < f(s^*)$}{
   $s^* = s'$\;
   }
 }
 \KwRet{$s^*$}

\caption{Initialisation phase}
\label{alg:phase1}
\end{algorithm}

After generating the initial solution, the iterative procedure starts by generating the set of neighbours of the current solutions. We consider a solution $s' \in X$ a neighbour of $s \in X$, whenever we can move an exam to a time slot in a different period. Evaluating the entire search space would be too time consuming. Instead, we first sort all periods by its contribution to the objective function. From the most conflictive period, we select the most conflictive time slot. Finally, we calculate all available time slots that we can swap with. In order to swap two time slots, the two affected exams (or sole affected exam when swapping to a time slot with no scheduled exam) must be able to be scheduled in the new time slots keeping the room type and capacity into account. This will ensure that no additional constraint 2 and 3 violations are introduced.

For each possible move, the objective score is calculated in order to rank all neighbours. The objective function for a solution $s$ can be defined as

\begin{equation}
    f(s) = \sum_{i=1}^{N} ( \sum_{j,k \in E_i}^{}p x_{jk})
\end{equation}
with $N$ the amount of periods, $E_i$ the set of exams scheduled to period $i$, $x_{jk}$ the number of common students between exams $j$ and $k$,
and $p$ the weight of conflicts.

After sorting all found neighbours by objective function, the best solution, that is not tabu or for which the aspiration criterion applies, is chosen. The aspiration criterion accepts solutions that are tabu as long as they are superior to $s^*$ (the best solution seen so far) If no moves are possible, we select the next most conflictive time slot until a suitable move has been found. When choosing the new solution, the tabu list is updated with the new tabu move consisting of the exam and period involved. Whenever the tabu list exceeds its maximum size, the oldest move is deleted, in order to allow that move again in future iterations.

This phase runs until a solution without conflicts has been found or the maximum number of iterations has been reached. For the latter case, the schedule will have conflicting exams during the same period with common students.

\subsubsection{Optimisation phase}

After the initialisation phase, we start the optimisation of the solution. Here the focus is including the soft constraints in the objective function to generate a solution that is the most optimal. The foundation of this phase is similar compared to the first phase with some distinct features. The overall flow can be seen in the pseudo code described in algorithm \ref{alg:phase2}.

\begin{algorithm}
\KwData{Solution s (the result of the initialisation phase)}
 $s^* = s$ (with $s^*$ the best solution seen so far)\;
 $k = 0$ (with k the number of iterations)\;
 \While{k < maximum number of iterations and $f(s^*) \neq 0$}{
  $k = k + 1$\;
  \uIf{$k\ \mathbf{mod}\ 3 \neq 0$}{
  \tcc{perform single moves of one exam to another time slot}
    Generate set of solutions $V^* \subseteq N(s, k)$\ (set of neighbours of s after single move)\;
  Sort $V^*$ by ascending $f(s')$ and select the best $s'$\;
   \ForEach{solution $s'$ in $V^*$}{
    \If{not $tabu(s')$ or $f(s') < f(s^*)$ (aspiration criterion)}{
        $s = s'$\;
        \textbf{break}\;
    }
   }
  }
  \Else{
    \tcc{swap entire period}
    Generate set of solutions $V^* \subseteq N(s, k)$\ (set of neighbours of s after swapping periods)\;
    Sort $V^*$ by ascending $f(s')$ and select the best $s'$\;
    $s = s'$\;
  }
  
   \If{$f(s') < f(s^*)$}{
   $s^* = s'$\;
   }
 }
 \KwRet{$s^*$}

\caption{Optimisation phase}
\label{alg:phase2}
\end{algorithm}


Firstly, the objective function is updated to take the distribution of exams into account. The objective of a solution is now defined as

\begin{equation}
    f(s) = \sum_{i=1}^{N} \sum_{j=1}^{N}(p_{|i-j|} \sum_{k \in E_i}^{}\sum_{l \in E_j}^{} x_{kl})
\end{equation}
with $N$ the amount of periods, $E_i$ the set of exams scheduled to period $i$, $x_{kl}$ the number of common students between exams $k$ and $l$, and $p_{|i-j|}$ the penalty for two exams scheduled at a distance of $|i-j|$ periods or days. While the objective function in the first phase only used the common students between exams during the same period, the new objective function will take all combination into account. By having a large penalty $p_0$ for exams scheduled on the same day, the hard constraint of having no exams on the same day for a student is still being prioritised. Additionally, placing exams at a further distance will now be preferred since the penalty $p_{x+1}$ will be smaller than the penalty $p_x$.

Secondly, two options are now available when generating the set of neighbours of the solution. The first option is a copy of the process in the initialisation phase where the set of neighbours of the solution is generated by performing single moves. Then the best solution that is either not tabu or that meets the aspiration criterion will be used. While this option might introduce constraint violations, the high $p_0$ will discourage and depending on the values used prevent this. This method will be alternated with a period permutation. Instead of moving a single exam, two entire periods will be swapped. Since the exams in the periods are not updated, this swap can not introduce any hard constraint violations. By moving entire periods, the objective function can change significantly in a single move. However, the amount of available moves is much lower and can be defined as
\begin{equation}
\begin{split}
   \text{\# possible moves}  & = \binom{N}{2}   \\
   & = \frac{N!}{2!(N-2)!}
\end{split}
\end{equation}

\subsection{Implementation: version 2}

After running experiments using version 1, some run time and result characteristics were observed. Version 2 diverts even more from the original algorithm in order to avoid these occurrences. Firstly, the objective function originally defined in the initialisation phase as   
\begin{equation}
    f(s) = \sum_{i=1}^{N} ( \sum_{j,k \in E_i}^{}p x_{jk})
\end{equation}
has an unwanted characteristic: it does not punish students with several exams on the same day extra. For exam schedules without a conflict free solution, this often results in students, who are enrolled in smaller exams, having a large number of exams on the same day. In order to avoid this behaviour, the objective function is updated to
\begin{equation}
    f(s) = \sum_{i=1}^{N} ( \sum_{s \in S(E_i)}^{}p^{c_{si} - 1})
\end{equation}
with $N$ the amount of periods, $E_i$ the set of exams scheduled to period $i$, $S(E_i)$ the set of students with more than 1 exam during $E_i$, $c_{si}$ the number of exams student $s$ has on period $i$, and $p$ the weight of conflicts. By placing $c_{si}$ in the exponent, having multiple exams on the same day is punished severely. 

Secondly, the way a tabu move is defined was changed. Originally, when moving an exam to a different period, the combination of the exam and period was used to define the tabu move. This resulted often resulted in the scenario that the same exam was continuously being moved. since this generally was one of the largest exams, it always has a large impact on the objective function and kept being determined as the most conflictive time slot. This behaviour can be avoided by having the exam as the sole identifier of the tabu move. By doing this, a single exam will not be moved as frequently.

\subsection{Implementation: version 3}

Version 1 and 2 both reduce the search space by only looking at the most conflictive time slot of the most conflictive period in order to determine the possible moves. Different time slots are only looked at when no moves are possible. This version steps away from that principle and continues evaluating the next most conflictive time slots their possible moves until a specified amount of possible moves have been found. Only then it selects the move generating the best next solution.

\subsection{Experiment}

For this case study, the data set to be used is provided by the FTI and FWET faculties of the University of Antwerp. The data consists of the exam information for the January and June exam period of 2021 and the schedule used. Even though the two faculties compose their own schedules, the problems are not independent. A section of the rooms are available for both faculties. As a consequence, the schedule has to be generated together and only then be split up. Otherwise a shared exam room could be double booked. The statistics for the data sets provided can be seen in table \ref{tab:data_set_sem1} and \ref{tab:data_set_sem2}. The fact that the scheduling problem for FTI and FWET is not independent can be seen here as well. The statistics for FTI and FWET combined do not equal the sum for the statistics of the faculties separated. The amount of time slots can be calculated as follows:
\begin{equation}
    \text{\# of time slots} = \text{\# Rooms} \times \text{\# Periods} \times \text{\# Exam times}  
\end{equation}
with there being two exam times, namely a morning and afternoon exam. For the January and June schedule there are respectively 20 and 24 periods.
\begin{table}[h]
	\caption{Data set statistics for January 2021}
	\label{tab:data_set_sem1}
	\centering
	\begin{tabular}{l c c c }
		\hline
		& \textbf{FTI} & \textbf{FWET} & \textbf{FTI+FWET} \\ \hline
		Students & 770 & 1041 & 1811 \\
		Exams & 185 & 360 & 536 \\
	    Rooms & 15 & 45 & 54 \\
        Time slots & 600 & 1800 & 2160 \\ \hline
	\end{tabular}
\end{table}

\begin{table}[h]
	\caption{Data set statistics for June 2021}
	\label{tab:data_set_sem2}
	\centering
	\begin{tabular}{l c c c }
		\hline
		& \textbf{FTI} & \textbf{FWET} & \textbf{FTI+FWET} \\ \hline
		Students & 782 & 1071 & 1852 \\
		Exams & 169 & 342 & 491 \\
	    Rooms & 15 & 45 & 54 \\
        Time slots & 720 & 2160 & 2592 \\ \hline
	\end{tabular}
\end{table}

Tabu search has a limited amount of hyperparameters compared to other optimisation methods. This aids in the parameter tuning process. The hyperparameters  available for the different versions can be seen in table \ref{tab:possible_parameters}.With MAX\_ITER\_OPTIMISATION and MAX\_ITER\_INITIALISATION defining the duration of the algorithm its run time, the main variables to set are the weights. Unlike P\_INITIALISATION being a single value, P\_OPTIMISATION is a list with the value at index $i$ containing the penalty for two exams at distance $i$. If there is no value at index $i$, the penalty is equal to 0.

\begin{table}[h]
	\caption{Tabu serach hyperparameters}
	\label{tab:possible_parameters}
	\centering
	\begin{tabular}{l c c}
		\hline
		& \textbf{Parameter} & \textbf{Description}  \\ \hline
        \textbf{All versions} & &\\ 
    
		& P\_INITIALISATION & weight of conflicts during initialisation phase \\
        & P\_OPTIMISATION  & weight of distance between exams\\
	    & MAX\_ITER\_INITIALISATION & max amount of iterations during initialisation phase \\
        & MAX\_ITER\_OPTIMISATION & max amount of iterations during optimisation phase\\ 
        \textbf{Version 3} & &\\ 
        & MAX\_MOVES & amount of moves to evaluate each iterations \\ 
        \hline
	\end{tabular}
\end{table}

\newpage
\section{Experimental Design}\label{sec:experiment}

For this case study, the data set to be used is provided by the Faculty of Applied Engineering and Faculty of Science of the University of Antwerp. The data consists of the exam information for the January and June exam period of 2021 and the actual schedule used. For some of the exams, students are split into different groups. This can happen because of room capacity or time constraints. We define for each group of students its own exam to be scheduled. Since these group divisions might not be optimal, further improvements can be made by adding the requirement to split large sets of students into smaller groups when needed.

Even though the two faculties compose their own schedules, the problems are not independent. A section of the rooms are available for both faculties. As a consequence, the schedule has to be generated together. Otherwise a shared exam room could be double booked. The statistics for the data sets provided can be seen in Table \ref{tab:data_set_sem1} and \ref{tab:data_set_sem2}. The fact that the scheduling problem for the Faculty of Applied Engineering and Faculty of Science is not independent can be seen here as well. The statistics for both faculties combined do not equal the sum for the statistics of the faculties separated. The amount of time slots can be calculated as follows:
\begin{equation}
    \text{\# of time slots} = \text{\# Rooms} \times \text{\# Periods} \times \text{\# Exam times}  
\end{equation}
with there being two exam times, namely a morning and afternoon exam. For the January and June schedule, there are respectively 20 and 24 periods.
\begin{table}[H]
	\caption{Data set statistics for January 2021}
	\label{tab:data_set_sem1}
	\centering
	\begin{tabular}{l c c c }
		\hline
		& \textbf{Faculty of Applied Engineering} & \textbf{Faculty of Science} & \textbf{Combined} \\ \hline
		Students & 770 & 1041 & 1811 \\
		Exams & 185 & 360 & 536 \\
	    Rooms & 15 & 45 & 54 \\
        Time slots & 600 & 1800 & 2160 \\ \hline
	\end{tabular}
\end{table}

\begin{table}[H]
	\caption{Data set statistics for June 2021}
	\label{tab:data_set_sem2}
	\centering
	\begin{tabular}{l c c c }
		\hline
		& \textbf{Faculty of Applied Engineering} & \textbf{Faculty of Science} & \textbf{Combined} \\ \hline
		Students & 782 & 1071 & 1852 \\
		Exams & 169 & 342 & 491 \\
	    Rooms & 15 & 45 & 54 \\
        Time slots & 720 & 2160 & 2592 \\ \hline
	\end{tabular}
\end{table}

Tables \ref{tab:workload_sem1} and \ref{tab:workload_sem2} show the workload of students for both data sets. The higher the amount of exams per student, the harder it becomes to create good timetables for students. 


\begin{table}[H]
	\caption{Student workload statistics for January 2021}
	\label{tab:workload_sem1}
	\centering
	\begin{tabular}{l c c c }
		\hline
		\textbf{\# exams per student}& \textbf{Faculty of Applied Engineering} & \textbf{Faculty of Science} & \textbf{Combined} \\ \hline
		Average  & 6.2 & 5.4 & 5.8 \\
		Median & 7 & 6 & 6 \\
	    Max & 11 & 11 & 11 \\
	\end{tabular}
\end{table}

\begin{table}[H]
	\caption{Student workload statistics for June 2021}
	\label{tab:workload_sem2}
	\centering
	\begin{tabular}{l c c c }
		\hline
		\textbf{\# exams per student}& \textbf{Faculty of Applied Engineering} & \textbf{Faculty of Science} & \textbf{Combined} \\ \hline
		Average  & 5.5 & 4.6 & 5.0 \\
		Median & 6 & 5 & 5 \\
	    Max & 10 & 11 & 11 \\
	\end{tabular}
\end{table}

Additionally, we compare the size and student workload statistics of the provided data set with the Toronto benchmark \cite{ceschia2022} (see Section \ref{benchmarks}). This comparison can be seen in Table \ref{tab:workload_compared}. W(avg) and W(max) detail the workload per student. These values, respectively, calculate the average and max amount of exams for every student. This shows that the size of the used data sets is competitive compared to the Toronto benchmark, especially when taking the workload into account.

\begin{table}[H]
	\caption{Student workload statistics compared between different data sets}
	\label{tab:workload_compared}
	\centering
	\begin{tabular}{l c c c c}
		\hline
		\textbf{Data set}& \textbf{Exams} & \textbf{Students} & \textbf{W(avg)} & \textbf{W(max)} \\ \hline
            \textbf{University of Antwerp} \\  \hline
		January 2021  & 536 & 1811 & 5.8 & 11 \\
		June 2021 & 491 & 1852 & 5.0 & 11 \\ \hline
              \textbf{Toronto benchmark instances} \\  \hline
		car91  & 678 & 16925 & 4.2 & 9 \\
		ear83 & 190 & 1125 & 7.2 & 7 \\
            hec92 & 81 & 2502 & 4.3 & 7 \\
            lse91 & 379 & 2627 & 4.2 & 8 \\
            rye93 & 485 & 9458 & 4.8 & 10 \\
            ute92 & 184 & 2672 & 4.4 & 6 \\
            yor83  & 81 & 940 & 6.42 & 14 \\ \hline
	\end{tabular}
\end{table}

In order to compare solutions, we propose the use of both qualitative and quantitative data. In section \ref{qualitative}, we discuss the use of qualitative data based on feedback from the university administration. Section \ref{quantitative} proposes several quantitative scoring metrics to be used. Section \ref{initial} discusses variance present in solutions when running the same analysis multiple times. Finally, section \ref{tuning} looks into optimising the hyper parameters used in the algorithm.

All analyses reported in the following subsections were conducted on a personal computer running an Intel Core i7-6700HQ processor with 16GB of RAM memory.

\subsection{Qualitative data} \label{qualitative}

The software used by the university administration is capable of exporting exam timetables into an Excel document. In order to provide the administrators with a familiar format, we export our generated timetables into the same Excel format. Every file corresponds to the entire timetable for one exam period and contains three data sheets, each visualising the timetable in a different manner. The first sheet details for every student the exams in which they are enrolled, and provides information on the scheduled exam such as the exam form (oral, written, PC, etc.), date, time (morning or afternoon), and room. An example for this sheet can be seen in Figure \ref{fig:sheet1}. 

\begin{figure}[H]
	\centering
	\includegraphics[width=0.7\textwidth]{images/excel/excel_sheet1.png} 
	\caption{Qualitative data: exam details per student}
	\label{fig:sheet1}
\end{figure}

The second sheet, shown in Figure \ref{fig:sheet2}, provides for every student the timetable of their study programmes. This is important because students can be enrolled in multiple study programmes. For example, a student can be enrolled in both an undergraduate and graduate degree when they are not enrolled in the model track. A student having more than one exam on the same day is visualised by a red block, highlighting the constraint violation.

\begin{figure}[H]
	\centering
	\includegraphics[width=0.9\textwidth]{images/excel/excel_sheet2.png} 
	\caption{Qualitative data: exam schedule per study programme}
	\label{fig:sheet2}
\end{figure}

The last sheet, shown in Figure \ref{fig:sheet3}, visualises for every student the overall timetable and the different exams. This provides an overview of all students their timetables. Additionally, it makes it possible to quickly identify which exams are conflicting for a student.

\begin{figure}[H]
	\centering
	\includegraphics[width=0.9\textwidth]{images/excel/excel_sheet3.png} 
	\caption{Qualitative data: exam schedule per student}
	\label{fig:sheet3}
\end{figure}

\subsection{Quantitative data} \label{quantitative}

The size of university timetabling schedules can make it hard to visually evaluate generated solutions. In order to objectively and consistently evaluate these solutions, we use several quantitative scoring metrics that can be used to compare and rank solutions.

First, the objective function is used in most research to compare solutions. Additionally, it is already used during the search to compare neighbours. This provides a consistent metric to rank solutions generated by both the algorithm and the university administration. However, this objective value does not allow to intuitively explain the quality of the solution. Additionally, having a second scoring metric will be required when trying to optimise the objective function. Since changing the weights of the function has an impact on its value, an independent measure is needed.

Thus, the average time between exams was considered as a timetable benchmark. However, this does not provide a suitable metric as the average can be easily skewed by outliers such as students having a low amount of exams. Instead the actual distribution of time between exams can be used. The importance of this metric became clear from the feedback provided by the university administrators. More notable, we use the percentage of times that students have exam conflicts, have 1 day between exams, 2 days, and finally 3+ days. While this does not result in a single metric, it allows for a good indication of the quality. This is because the administration attempts to provide at least 2 days between exams for the model track, and preferably 3 or more.

\subsection{Variance in results} \label{initial}

In order to compare results between different runs, we must first look at how much variance there is when running the same analysis multiple times. We look at the impact the initial solution has on the optimisation phase and the final solution. As explained in Section \ref{initialisation}, the initial solution takes all hard constraints into account except the exam conflicts between solutions. However, every exam is assigned a random time slot as long as its room type and capacity meet the requirements.

Table \ref{fig:multiple_runs} shows the objective function for different analysis runs of version 2 with an identical set of hyper parameters. While there is some variance present, the amount of spread remains limited. Additionally, the impact on the exam distribution can be seen in Table \ref{tab:multiple_runs}. Even though some differences in distribution are visible, the standard deviation remains low. This makes that the results of our analyses can be considered robust, meaning the majority of duplicate runs will be very similar. Nonetheless, for all comparisons, we will select the best run out of several in order to provide the most accurate comparison.

\begin{figure}[H]
	\centering
	\includegraphics[width=0.75\textwidth]{images/initial/multiple_runs.png} 
	\caption{Objective function when running the same analysis multiple times}
	\label{fig:multiple_runs}
\end{figure}

\begin{table}[h]
	\caption{Exam distribution between duplicate runs}
	\label{tab:multiple_runs}
	\centering
	\begin{tabular}{c c c c c c}
		\hline
  	\textbf{Run}	&
   \textbf{0 days \% } &
    \textbf{1 day \% } & 
    \textbf{2 days \% } &
    \textbf{3 days \% } & 
    \textbf{4+ days \%}\\ \hline
    Run 1 & 0.1\% & 8.4\% & 8.5\% & 22.5\% & 60.5\% \\
    Run 2 & 0.0\% & 8.6\% & 9.1\% & 20.3\% & 62.0\% \\
    Run 3 & 0.0\% & 8.0\% & 8.7\% & 22.3\% & 61.0\% \\
    Run 4 & 0.0\% & 9.2\% & 7.1\% & 23.1\% & 60.6\% \\
    Run 5& 0.0\% & 9.2\% & 9.7\% & 20.9\% & 60.2\% \\
    Run 6 & 0.0\% & 8.0\% & 7.4\% & 22.6\% & 62.0\% \\ \hline
    
    Average& 0.0\% & 8.6\% & 8.4\% & 22.0\% & 61.1\%\\
    Median & 0.0\% & 8.6\% & 8.6\% & 22.4\% & 60.8\% \\\hline
    
    Standard Deviation & 0.0\% & 0.5\% & 1.0\% & 1.1\% & 0.78\%\\
        \hline
	\end{tabular}
\end{table}

\subsection{Parameter tuning} \label{tuning}

Parameter tuning is important in order to optimise the performance of search methods. The different hyper parameters  available for the algorithm versions can be seen in Table \ref{tab:possible_parameters}. With MAX\_ITER\_OPTIMISATION and MAX\_ITER\_INITIALISATION defining the duration of the algorithm run time, the main variables to set are the weights. Unlike P\_INITIALISATION being a single value, P\_OPTIMISATION is a list with the value at index $i$ corresponding to the penalty for two exams at distance $i$. If there is no value at index $i$, the penalty is equal to 0.

\begin{table}[h]
	\caption{Tabu search hyper parameters}
	\label{tab:possible_parameters}
	\centering
	\begin{tabular}{l c c}
		\hline
		& \textbf{Parameter} & \textbf{Description}  \\ \hline
        \textbf{All versions} & &\\ 
		& P\_INITIALISATION & weight of conflicts during initialisation phase \\
        & P\_OPTIMISATION  & weight of distance between exams\\
	    & MAX\_ITER\_INITIALISATION & max amount of iterations during initialisation phase \\
        & MAX\_ITER\_OPTIMISATION & max amount of iterations during optimisation phase\\ 
        & TABU\_LIST\_SIZE & size of the tabu list used\\
        \textbf{Version 3} & &\\ 
        & MAX\_MOVES & amount of moves to evaluate each iteration \\ 
        \hline
	\end{tabular}
\end{table}





\subsubsection{TABU\_LIST\_SIZE }

The TABU\_LIST\_SIZE hyper parameter is partly responsible for determining the balance between exploration and exploitation by blocking previously used moves to be applied. Larger tabu lists encourage exploration, which aids the search in escaping local optima. However, if the size is too large, the amount of exploitation is minimised and promising solutions may not be further optimised. Hence, the tabu list size can be a crucial component in the search's performance.

Originally, the size of the tabu list was calculated by Alvarez-Valdes et al. as:

\begin{equation}
    \text{Tabu list size} = \lfloor \sqrt{\text{\# Exams} * \text{\# Periods}}\rfloor
\end{equation}

In order to tune the size of the tabu list (Equation \ref{eq:list}), we add a weight $w$ to it, resulting in:

\begin{equation}
    \text{Tabu list size} = \lfloor\text{w} * \sqrt{\text{\# Exams} * \text{\# Periods}}\rfloor
\end{equation}

By having the weight either smaller or larger than 1, we can perform an analysis with a smaller or larger tabu list size. For the tuning, the weights $w$ were set to 0.1, 0.25, 0.5, 0.75, 1, 2, and 3, resulting in tabu list sizes of 6, 15, 31, 47, 63, 126, and 189, respectively. The impact on the objective by changing the tabu list size can be seen in both Figure \ref{fig:tuning_tabu} and Table \ref{tab:tuning_tabu_obj}. Based on the objective function, using a weight of 0.5  has the most positive impact on the objective of the solutions. This means that the tabu list size to be used is half of the proposed size by Alvarez-Valdes et al. Table \ref{tab:tuning_tabu_distr} explains this decrease in objective by showing a lower amount of occurrences where a student has only 1 day between exams compared to the other results. Every 'x days \%' column indicates the percentage of occurrences where a student has x days between exams. 

\begin{figure}[H]
	\centering
	\includegraphics[width=0.75\textwidth]{images/tuning/tabu_list.png} 
	\caption{Objective function when testing different TABU\_LIST\_SIZE \\ values for the Faculty of Science June data set}
	\label{fig:tuning_tabu}
\end{figure}

\begin{table}[h]
	\caption{Effect of tabu list size on objective functions}
	\label{tab:tuning_tabu_obj}
	\centering
	\begin{tabular}{l c c c }
		\hline
  	\textbf{Weight $w$}	                & 
     	\textbf{Tabu list size}	                & 
    \textbf{Objective} & 
    \textbf{Objective difference vs $w$ = 1} \\ \hline
        0.1 & 6 & 71738 & -2.3\%  \\ 
        0.25& 15 & 71908 & -2.1\%  \\ 
        0.50 & 31 & 66764 & -9.1\%  \\ 
        0.75 & 47 & 77798 & 5.9\% \\
        1 & 63 &73450 & 0.0\% \\
        2 & 126 & 85728 & 16.7\% \\
        3 & 189 & 79824 & 8.7\% \\
        \hline
	\end{tabular}
\end{table}

\begin{table}[h]
	\caption{Effect of tabu list size on exam distribution}
	\label{tab:tuning_tabu_distr}
	\centering
	\begin{tabular}{c c c c c c c}
		\hline
  	\textbf{Weight $w$}	&
   \textbf{Tabu list size} &
   \textbf{0 days \% } &
    \textbf{1 day \% } & 
    \textbf{2 days \% } &
    \textbf{3 days \% } & 
    \textbf{4+ days \%}\\ \hline
    0.1 & 6 & 0.0\% & 6.2\% & 11.0\% & 19.5\% & 63.3\% \\
    0.25 & 15 & 0.0\% & 6.1\% & 10.6\% & 20.3\% & 63.0\% \\
    0.5 & 31 & 0.0\% & 5.0\% & 12.8\% & 21.3\% & 60.9\% \\
    0.75 & 47 & 0.0\% & 6.7\% & 12.8\% & 21.4\% & 59.1\% \\
    1 & 63 & 0.0\% & 6.0\% & 14.0\% & 14.0\% & 66.0\% \\
    2 &  126 & 0.0\% & 8.3\% & 14.2\% & 20.2\% & 57.3\% \\
    3 &  189 & 0.0\% & 6.7\% & 12.7\% & 24.8\% & 55.8\% \\
        \hline
	\end{tabular}
\end{table}

\subsubsection{P\_INITIALISATION \& P\_OPTIMISATION}

Since the initialisation phase is successful in obtaining feasible timetables (Section \ref{phase_init}), we do not further explore fine tuning P\_INITIALISATION. The optimisation phase is responsible for improving the distribution to result in a more optimal timetable. Because of this, we look into tuning the P\_OPTIMISATION weights.

For P\_OPTIMISATION, we look at the results from four sets of weights. Every weight $w_x$ penalises exams with $x$ days between the two exams. If no weight is present for $w_x$, the weight is set to 0.  First, we test the set of weights described by Alvarez-Valdes et al. namely $w_0$ = 3000, $w_1$ = 100, $w_2$ = 20, $w_0$ = 5, $w_4$ = 3, and $w_5$ = 1. Second, the weights 16, 8, 4, 2, and 1 are used for the Carter benchmarks \cite{carter1996}. Finally, we add two sets of weights that have a higher penalty for $w_1$ and $w_2$ than the one for $w_0$ in order to force the distribution to change after the initialisation phase. The weights used for this are [100, 5000, 2000, 50, 30, 5] and [250, 3000, 1000, 50, 30, 5]. Since a change in weights will impact the objective function, the exam distribution has to be used in order to determine the best set of weights.

\begin{table}[h]
	\caption{Effect of objective weights on exam distribution}
	\label{tab:weights_distr}
	\centering
	\begin{tabular}{c c c c c c}
		\hline
  	\textbf{Weights $w_x$}	&
   \textbf{0 days \% } &
    \textbf{1 day \% } & 
    \textbf{2 days \% } &
    \textbf{3 days \% } & 
    \textbf{4+ days \%}\\ \hline
    3000, 100, 20, 5, 3, 1 & 0.0\% &  8.8\% & 12.3\% & 18.1\% & 60.8\% \\
    16, 8, 4, 2, 1 & 0.0\% & 7.4\% & 13.7\% & 18.3\% & 60.6\% \\
    100, 5000, 2000, 50, 30, 5 & 0.0\% & 7.6\% & 9.6\% & 20.2\% & 62.6\% \\
    250, 3000, 1000, 50, 30, 5 & 0.0\% &  7.9\% & 11.9\% & 19.3\% & 60.9\% \\
        \hline%
	\end{tabular}
\end{table}

Table \ref{tab:weights_distr} shows the resulting distribution after a maximum of 1000 iterations (fewer if no improved solution has been found after 200 iterations). While the weights of the Carter benchmark (16, 8, 4, 2, and 1) provide the lowest amount of exams occurrences with a single day between them, the third set of weights provides a better overall distribution. Even though it has a slightly higher percentage of occurrences for exams at distance 1, it significantly lowers the amount of exams at a distance of 2 days compared to the other weights.


\subsubsection{MAX\_ITER\_OPTIMISATION \& MAX\_MOVES}

Thompson and Dowsland \cite{thompson1996} showcase that longer searches are able to generate better solutions. This can be explained by a longer search resulting in more solutions being visited in the search space. This results in a larger possibility of encountering good solutions. The hyper parameters MAX\_ITER\_OPTIMISATION \& MAX\_MOVES both impact the duration of the search. While MAX\_ITER\_OPTIMISATION increases the amount of iterations for the search to end, MAX\_MOVES increases the amount of time spent on each iteration. 

Previous analyses have shown that the search is generally able to converge in fewer than 500 iterations to the point where no improvement is found for at least 200 iterations. This is visible in Figure \ref{fig:tuning_tabu} where all runs end with a flat lining objective function. While better solutions may still be found due to the search currently being stuck in a local optimum, terminating the solution after a cut off point is considered acceptable. Because of this, the importance of having a high value for MAX\_ITER\_OPTIMISATION is reduced.

The impact of increasing the MAX\_MOVES parameter is tested on both the execution time and objective function. The values tested for MAX\_MOVES are 250, 500, 750, 1000, and 1500. Figure \ref{fig:tuning_execution} shows that increasing MAX\_MOVES generally also lowers the objective function. 

\begin{figure}[H]
	\centering
	\includegraphics[width=0.75\textwidth]{images/tuning/max_moves_objective.png} 
	\caption{Objective function when testing different MAX\_MOVES \\ values for the combined June data set}
	\label{fig:tuning_objective}
\end{figure}

Additionally, Figure \ref{fig:tuning_execution} visualises the relation of the value for MAX\_MOVES and the execution time. It shows that increasing MAX\_MOVES also results in a longer execution time per iteration. Since a MAX\_MOVES value of 250 results in one of the best objective function at the lowest time penalty, we will use this value for further analyses. 

\begin{figure}[H]
	\centering
	\includegraphics[width=0.75\textwidth]{images/tuning/max_moves_execution.png} 
	\caption{Execution time when testing different MAX\_MOVES \\values for the combined June data set}
	\label{fig:tuning_execution}
\end{figure}


\newpage
% !TEX root = ../CedricDe Schepper2023_Thesis.tex

\section{Results / Evaluation}\label{sec:results}

In order to evaluate the results from the analyses, we have to verify whether the phases perform their objective as described. First, does the initialisation phase succeed in generating a feasible solution, namely one without or with a minimal number of hard constraint violations? Second, does the optimisation phase succeed in significantly improving the exam distribution? These results can then be compared against the reference solutions to determine the feasibility of using automated timetabling by the University of Antwerp.

\subsection{Reference solutions}

The actual exam schedules, manually created by the administration of the University of Antwerp, can be used as a baseline to compare the generated schedules. These exam distributions can be seen in Figures \ref{fig:manual_sem1} and \ref{fig:manual_sem2}. Figure \ref{fig:manual_combined} shows the exam distribution when combining the schedule of both the Faculty of Applied engineering and the Faculty of  Science. Most notably, the manual schedules have a focus on keeping the amount of same day exams to a minimum and having 2 or more days in between exams as much as possible. This distribution is especially visible in the schedules for the Faculty of Applied Engineering. In order to be considered a superior solution, automatically generated schedules will have to minimise the amount of exams with fewer than 2 days in between, with a focus on same day exam violations.

Some additional observations can be made. First, it can be seen that the schedule for the faculty of Science contains a high amount of exam conflicts. Second, the exam distribution is much worse for the January exam period compared to the one for June. This can clearly be seen in the combined exam distribution shown in Table \ref{fig:manual_combined}. This can be mostly be explained by January having more exams within a smaller period. For the January exam schedule, 536 exams must be assigned to time slots in 20 periods. This corresponds to an average of 27 exams per period. Meanwhile, the June exam period consists of 491 exams, to be scheduled in 24 periods. This results in a smaller average of 21 exams per period.

The graphs shown in Figures \ref{fig:manual_sem1}, \ref{fig:manual_sem2}, and \ref{fig:manual_combined} are generated by calculating the distance between each exam for every student, with the exams being sorted on exam date. For example, a student A with exams on day 1, 5, and 7 will contribute the distance between day 1 and 5, as well as the distance between day 5 and 7. This results in the student contributing to the $x = 4$ and $x = 2$ bar, respectively. We can also look at the case for a student B with conflicting exams. This can be showcased using an identical schedule compared to student A, with the addition of a conflicting exam on day 5. This results in the distances being calculated using the sequence of 1, 5, 5, and 7. Because of this, student B will contribute to the $x = 4$, $x = 0$, and $x = 2$ bar, respectively.

\begin{figure}[H]
  \centering
  \subfloat[FTI]{\includegraphics[width=0.5\textwidth]{images/manual/fti_sem1_percentages_manual.png}}
  \hfill
  \subfloat[FWET]{\includegraphics[width=0.5\textwidth]{images/manual/fwet_sem1_percentages_manual.png}}
  \caption{Manual exam distribution for January 2021}
  \label{fig:manual_sem1}
\end{figure}

\begin{figure}[H]
  \centering
  \subfloat[FTI]{\includegraphics[width=0.5\textwidth]{images/manual/fti_sem2_percentages_manual.png}}
  \hfill
  \subfloat[FWET]{\includegraphics[width=0.5\textwidth]{images/manual/fwet_sem2_percentages_manual.png}}
  \caption{Manual exam distribution for June 2021}
  \label{fig:manual_sem2}
\end{figure}

\begin{figure}[H]
  \centering
  \subfloat[January 2021]{\includegraphics[width=0.5\textwidth]{images/manual/existing_combined_sem1_percentages.png}}
  \hfill
  \subfloat[June 2021]{\includegraphics[width=0.5\textwidth]{images/manual/existing_combined_sem2_percentages.png}}
  \caption{Manual exam distribution for FTI and FWET combined}
  \label{fig:manual_combined}
\end{figure}


\subsection{Initialisation phase} \label{phase_init}

In order to quantify the feasibility of a solution, it is possible to look at either the objective function or the amount of hard constraint violations. Since the objective function was adapted to contain exponents and the conflict weight having a large impact on the size of the objective, looking at the amount of hard constraint violations can provide a better perspective. By plotting the violations per iteration, the progress made by the initialisation phase can be visualised. Figure \ref{fig:violations} shows the amount of exam conflicts for students. The exam conflicts are guaranteed to be the only hard constraint violations present because of how the initialisation phase generates the initial solution. There, exams are only assigned to time slots and rooms if they do not violate constraint 2, 3, and 4. This means that the room capacity, type, and availability to faculty is already taken into account. Because of this, the amount of conflicts shown is the amount of constraint 5 violations, namely the amount of times a student has multiple exams on the same day. Since the amount of conflicts reduces to 0, it proves that the initialisation phase is able to generate a solution for the Faculty of Applied Engineering and the Faculty of Science without hard constraint violations. 

It shows that the initialisation phase is able to reach a feasible solution for both data sets. Both the amount of conflicts and the execution time is higher for the January data set compared to the results for June. This can be explained by the higher workload per student for January combined with the lower amount of periods (Section \ref{sec:experiment}). This makes the amount of hard constraint violations during the initialisation solution higher, which thus requires more iterations to resolve.

\begin{figure}[H]
  \centering
  \subfloat[January 2021]{\includegraphics[width=0.5\textwidth]{images/init/sem_1_conflicts.png}}
  \hfill
  \subfloat[June 2021]{\includegraphics[width=0.5\textwidth]{images/init/sem_2_conflicts.png}}
  \caption{Hard constraint violations during initialisation phase}
  \label{fig:violations}
\end{figure}

Figure \ref{fig:phase_comparison} displays the execution time of the initialisation phase compared to the optimisation phase for the same amount of iterations. It shows that the initialisation phase is highly effective at finding a feasible solution, taking significantly less time compared to running the optimisation phase.

 
\begin{figure}[H]
  \centering
  \subfloat[January 2021]{\includegraphics[width=0.5\textwidth]{images/init/sem1_phase_comparison.png}}
  \hfill
  \subfloat[June 2021]{\includegraphics[width=0.5\textwidth]{images/init/sem2_phase_comparison.png}}
  \caption{Execution time comparison between initialisation and optimisation phase}
  \label{fig:phase_comparison}
\end{figure}

However, Figure \ref{fig:init} shows that no attention was given to the distribution. This can be deduced by the presence of a high percentage of exams after a single day. The exam distribution also confirms that there are no overlapping exams present after the initialisation phase.

\begin{figure}[H]
  \centering
  \subfloat[January 2021]{\includegraphics[width=0.5\textwidth]{images/init/sem1.png}}
  \hfill
  \subfloat[June 2021]{\includegraphics[width=0.5\textwidth]{images/init/sem2.png}}
  \caption{Exam distribution after the initialisation phase}
  \label{fig:init}
\end{figure}

\subsection{Optimisation phase}

While we have shown that tabu search can efficiently generate a feasible solution from the provided data set, the exam distribution will determine whether the automated schedules are superior in quality compared to the manual timetables provided by the university's administration.

First, we look at which Tabu Search version is able to generate the best results. Afterwards, we'll compare the generated timetables with the reference solutions.

\subsubsection{Comparison between Tabu Search versions}

In order to compare the different versions as explained in Section \ref{sec:method}, we look at the execution time and exam distribution.  Figure \ref{fig:version_comparison} shows the execution time when running the search algorithm versions for the two exam periods. For version 1, it shows that the execution time per iteration remains constant and is significantly lower compared to both version 2 and 3. Additionally, it can be seen that version 2 is faster than version 3. However, the difference between them depends on the data set used. This difference between version 1 and the other two versions can be explained by the change in what describes a tabu move. Instead of considering a move to be a combination of the exam and period as used in version 1, version 2 and 3 identify a move solely by the exam. Because of this change, a lot of moves that were considered not tabu in version 1 because of a different period targeted, are seen as tabu in the other versions. This makes that these versions often have to check significantly more moves, resulting in a longer execution time. This can also explain why version 2 and 3 have a similar execution time for the June data set. Whenever the search is converging, it can become difficult to find a solution that is considered superior to the one already known. This makes that the search will have to consider a high amount of moves before finding a move, that can be successfully applied. Because of this, the number of moves explored during the search in version 2 can be similar to the number explored in version 3. This is especially true when setting a smaller MAX\_MOVES parameter. 

\begin{figure}[H]
  \centering
  \subfloat[January 2021]{\includegraphics[width=0.5\textwidth]{images/results/versions/sem1_execution_times.png}}
  \hfill
  \subfloat[June 2021]{\includegraphics[width=0.5\textwidth]{images/results/versions/sem2_execution_times.png}}
  \caption{Execution time comparison between the different Tabu Search versions}
  \label{fig:version_comparison}
\end{figure}


Tables \ref{tab:version_comparison_sem1} and \ref{tab:version_comparison_sem2} detail the exam distribution after consolidating the best results from multiple runs for each version. Several conclusions can be made from these results. First, version 1 clearly performs worse compared to the other versions for both data sets. This can especially be seen for the June data set where 23.9\% of the time a student has two or fewer days between exams. This is only 15.2\% and 16.3\% for version 2 and version 3, respectively. Second, the difference in performance between version 2 and 3 is minimal. Version 2 performing equally well or better than version 3 can again be explained by the MAX\_MOVES parameter losing its importance during long runs. The reason behind this is due to the tabu move definition, which also affected the execution time. This means that we will focus on the results by version 2 in future analyses.

Finally, the difference in quality between January and June is quite notable. For January, the best result with version 2 has two or fewer days between exams 27.9\%. This is significantly worse than the best result for June, where that only occurs 15.3\% of the time.

\begin{table}[h]
	\caption{Exam distribution comparison between the different Tabu Search versions for January 2021}
	\label{tab:version_comparison_sem1}
	\centering
	\begin{tabular}{c c c c c c}
		\hline
  	\textbf{Version}	&
   \textbf{0 days \% } &
    \textbf{1 day \% } & 
    \textbf{2 days \% } &
    \textbf{3 days \% } & 
    \textbf{4+ days \%}\\ \hline
    Version 1 & 1.9\% & 16.7\% & 14.7\% & 18.4\% & 48.3\% \\
    Version 2 & 0.1\% & 14.5\% & 13.3\% & 20.6\% & 51.5\% \\
    Version 3 & 0.1\% & 15.3\% & 13.1\% & 24.6\% & 46.9\% \\
        \hline
	\end{tabular}
\end{table}

\begin{table}[h]
	\caption{Exam distribution comparison between the different Tabu Search versions for June 2021}
	\label{tab:version_comparison_sem2}
	\centering
	\begin{tabular}{c c c c c c}
		\hline
  	\textbf{Version}	&
   \textbf{0 days \% } &
    \textbf{1 day \% } & 
    \textbf{2 days \% } &
    \textbf{3 days \% } & 
    \textbf{4+ days \%}\\ \hline
    Version 1 & 0.2\% & 11.5\% & 12.2\% & 17.2\% & 58.9\% \\
    Version 2 & 0.0\% & 6.8\% & 8.4\% & 24.3\% & 60.5\% \\
    Version 3 & 0.3\% & 7.8\% & 8.2\% & 22.0\% & 61.7\% \\
        \hline
	\end{tabular}
\end{table}
 \label{subsec:versions}
\subsubsection{Comparison between generated and reference solutions}

The results from Section \ref{subsec:versions} allow us to determine the best version and with it the best generated solution. However, we must compare the generated solutions with the reference solutions in order to answer the question whether the automated solutions outperform the manual timetables. Figures \ref{fig:sem1_comparison} and \ref{fig:sem2_comparison} provide the exam distribution of the generated solutions next to the reference solutions. 

The generated solution for January 2021 has two or fewer days between exams 27.9\% of time, while that is only 12\% for the reference solution.  Although a smaller difference, the same can be observed for June with a percentage of 15.4\% and 11.9\% for the generated and reference solution, respectively. However, the distribution looks very different with the generated solutions having nearly no exam conflicts. This does occur for the reference solution, for almost 3\% of the time in the solutions for both data sets.


\begin{figure}[H]
  \centering
  \subfloat[Reference solution]{\includegraphics[width=0.5\textwidth]{images/manual/existing_combined_sem1_percentages.png}}
  \hfill
  \subfloat[Generated solution]{\includegraphics[width=0.5\textwidth]{images/results/final/sem1_generated.png}}
  \caption{Exam distribution comparison for January 2021}
  \label{fig:sem1_comparison}
\end{figure}

\begin{figure}[H]
  \centering
  \subfloat[Reference solution]{\includegraphics[width=0.5\textwidth]{images/manual/existing_combined_sem2_percentages.png}}
  \hfill
  \subfloat[Generated solution]{\includegraphics[width=0.5\textwidth]{images/results/final/sem2_generated.png}}
  \caption{Exam distribution comparison for June 2021}
  \label{fig:sem2_comparison}
\end{figure}

\subsubsection{Qualitative feedback on the generated solutions}

In addition to the quantitative comparison between the generated and reference solutions, we also reviewed the obtained timetables with the university administrators. This was done by providing them with the timetables in Excel format, as detailed in Section \ref{qualitative}, and with the exam distribution figures visible in Figures 
\ref{fig:generated_sem1} and \ref{fig:generated_sem2}. These figures provide a more detailed view on the exam distribution by splitting the two faculties. 

\begin{figure}[H]
  \centering
  \subfloat[Faculty of Applied Engineering]{\includegraphics[width=0.5\textwidth]{images/results/final/fti_sem1.png}}
  \hfill
  \subfloat[Faculty of Science]{\includegraphics[width=0.5\textwidth]{images/results/final/fwet_sem1.png}}
  \caption{Generated exam distribution for January 2021}
  \label{fig:generated_sem1}
\end{figure}

\begin{figure}[H]
  \centering
  \subfloat[Faculty of Applied Engineering]{\includegraphics[width=0.5\textwidth]{images/results/final/fti_sem2.png}}
  \hfill
  \subfloat[Faculty of Science]{\includegraphics[width=0.5\textwidth]{images/results/final/fwet_sem2.png}}
  \caption{Generated exam distribution for June 2021}
  \label{fig:generated_sem2}
\end{figure}

The university administrators were able to add insights that are not visible in the quantitative results. First, they note that the search method correctly takes into account exams that belong to different enrolment years within the model track. For example, some of the students that have two exams scheduled on consecutive days, only have this for exams that are not part of the same year. While this negatively impacts the distribution statistics, its actual impact is minimal because no priority is given to two exams that are not within the same year for the model track. However, the model track has no extra weight when assigning exams to time slots, resulting in cases where a model track student only has a single day between exams. The university's administration strongly attempts to prevent this, often at the cost of the distribution for non-model track students. The presence of these cases reduces the feasibility of using the generated solutions.

Second, the university administrators confirm that the search algorithm prioritises exams with a large amount of students enrolled over small exams. Since we consider all students that are enrolled for every exam, the penalties incurred by poorly scheduled large exams have a significant impact on the objective function. 

Finally, they conclude that while the obtained distributions are not currently sufficient as final solutions, the generated solutions can act as an initial baseline. This would reduce the time spent on generating an initial solution. The solution could then be further optimised manually.




\newpage
% !TEX root = ../CedricDe Schepper2023_Thesis.tex

\section{Threats to Validity}\label{sec:threats}

In this section, we discuss the threats to the validity of the experiments performed. The presence of these threats could undermine the trustworthiness of the obtained results, and negatively affect the quality of this thesis. For every threat, we look at how we mitigated it.


% Figure~\ref{fig:threats} shows a diagram of the experiment principles and where lies each threat.
% Please do not use this image on the paper/thesis.
% Reviewers are supposed to know where the threats lie, the figure is for students to better understand the threats.
% \begin{figure}[H]
% 	\centering
% 	\includegraphics[width=0.75\textwidth]{images/threats-to-validity.png} 
% 	\caption{Experimental principles diagram showing threats. (1) Conclusion; (2) Internal; (3) Construct; and (4) External.}
% 	\label{fig:threats}
% \end{figure}


\subsection{Conclusion validity}

Conclusion validity refers to the reliability of the conclusions drawn from the generated results. Since only two data sets were used during the experiments, this could pose a threat to the validity of those results. However, the characteristics and size of the data set help in mitigating this threat. Both data sets consists of data from two faculties, and add up to a notable size. The size can be compared to the data set size and student workload statistics for the Toronto benchmark \cite{ceschia2022} (see Section \ref{benchmarks}). These comparison can be seen in Table \ref{tab:workload_compared}. W(avg) and W(max) detail the workload per student. These values, respectively, calculate the average and max amount of exams for every student. This shows that the size of the used data sets is competitive compared to the Toronto benchmark, especially when taking the workload into account.

\begin{table}[H]
	\caption{Student workload statistics compared between different data sets}
	\label{tab:workload_compared}
	\centering
	\begin{tabular}{l c c c c}
		\hline
		\textbf{Data set}& \textbf{Exams} & \textbf{Students} & \textbf{W(avg)} & \textbf{W(max)} \\ \hline
            \textbf{\acrfull{ua}} \\  \hline
		January 2021  & 536 & 1811 & 5.8 & 11 \\
		June 2021 & 491 & 1852 & 5.0 & 11 \\ \hline
              \textbf{Toronto benchmark instances} \\  \hline
		car91  & 678 & 16925 & 4.2 & 9 \\
		ear83 & 190 & 1125 & 7.2 & 7 \\
            hec92 & 81 & 2502 & 4.3 & 7 \\
            lse91 & 379 & 2627 & 4.2 & 8 \\
            rye93 & 485 & 9458 & 4.8 & 10 \\
            ute92 & 184 & 2672 & 4.4 & 6 \\
            yor83  & 81 & 940 & 6.42 & 14 \\ \hline
	\end{tabular}
\end{table}


% Conclusion validity is the degree to which conclusions we reach about relationships in our data are reasonable.
% \begin{itemize}
%    \item  It affects the ability to draw correct conclusion about relations between treatment and outcome. \\
% + Choice of statistical tests \\
% + Choice of sample size \\
% + Measurement of the experiment
%   \item Low Statistical Power \\
% + If power is low, there is a high risk that an erroneous conclusion is drawn.
% \end{itemize}

\subsection{Internal validity}

Threats to the internal validity of experiments risk impacting the trustworthiness of the results observed. These threats were mitigated in several ways. Firstly, the randomness in generating the initial solution helps preventing bias from infiltrating the results. Additionally, all experiments were run multiple times in order to verify that the results obtained were robust and to remove outliers. Overall, it was observed that the results after all runs produced similar results.


% Are there any other factors that may affect the results?
% \begin{itemize}
%   \item Were phenomena observed under special conditions \\
% + in the lab, close to a deadline, company risked bankruptcy, … \\
% + major turnover in team, contributors changed (open-source), …
%   \item Similar observations repeated over time (learning effects)
%   \item Correlation does not imply causation.
% \end{itemize}

\subsection{Construct validity}

The construct validity threat focuses on how accurately the scoring metrics assess the quality of solutions. The objective function and exam distribution are the two main scoring metrics used. The objective function is based on the objective function by Alvarez-Valdes et al. \cite{alvarez1997}. Additionally, the use of the exam distribution metric has been discussed and accepted as a valid metric by the University's administration in order to rate the quality of solutions.

\subsection{External validity}

The external validity threat refers to the generalisability of the findings, namely to what extend the findings can be applied to other situations. The use case presented by data and constraints of the University of Antwerp timetabling problem did not fit within any of the existing benchmarks. This could create a potential threat to the external validity of our experiments. However, the improvements applied to the original algorithm are generic in nature. For example, Version 3, as detailed in Section \ref{version3}, expands the amount of search space explored during every iteration. This expansion does not take into account any specificities of our use case and can be applied to all timetabling problems.

% A possible threat to the external validity of our experiments can be the 
% To what extent can the findings be generalized?
% \begin{itemize}
%   \item Does it apply to other languages? Other sizes? Other domains? Other systems?
%   \item Background \& education of participants
%   \item Simplicity \& scale of the team \\
% + small teams \& flexible roles vs. large organizations \& fixed roles
% \end{itemize}

\newpage
% !TEX root = ../CedricDe Schepper2023_Thesis.tex

\section{Conclusions}\label{sec:conclusion}

 In this thesis, we investigated how an implementation of the Tabu Search method compares to manual generated examination timetables for the University of Antwerp. Due to constraints such as the requirement for different exam room types and the notion of model track students, we were unable to copy existing implementations that proved successful in recent benchmarks. 

 The analyses performed show that the Tabu Search implementations were able to generate solutions without any exam conflicts. In order to provide a better final exam distribution, we successfully improved the Tabu Search implementation, proposed by Alvarez-Valdes et al.\cite{alvarez1997}. These changed reduced the percentage of occurrences where a student has two or fewer days between exams. For both the January and June data set, this distribution improved from 33.3\% to 27.3\%, and from 23.9\% to 15.2\%, respectively.
 
When comparing the generated solutions to the reference solutions, the proposed implementation was able to reduce the amount of exam conflicts from nearly 3\% to zero or close to zero. However, the change in objective function was not able to effectively change Constraint 5 into a soft constraint, in order to improve the overall distribution. For the January 2021 data set, 27.9\% of the time a student had two or fewer days between exams, while the reference solution was superior with only 12\%. The same was visible for June 2021, with 15.4\% versus the 11.9\% of the reference solution. From feedback by the university's administration, we gathered that the solution does not succeed in providing the wanted distribution. However, they did agree that it can act as an initial baseline.

In conclusion, we laid the foundation for further research by formalising the requirements enforced by the university's administration into hard and soft constraints. Additionally, we provided an algorithm, that is capable of creating timetables without exam conflicts. These timetables can then be improved manually. This combination would reduce the amount of time spent while keeping the flexibility observed during manual creation.

Based on the scheduling problem defined, future research could investigate the performance when implementing different search methods. Additionally, the search problem can be further expanded by increasing the amount of constraints present. For example, by adding constraints unique to certain exams, such as an exam requesting a specific room.



% \begin{itemize}
% \item \textbf{Summary of Results.} 
%    In a paper/thesis, we probably have many pages in previous sections presenting results. 
%    Now in the conclusion, it is time to put the most important results here for the reader. 
%    Especially research with measurable results, we highlight the numbers here. 
% \item \textbf{Main Findings / Conclusions.} 
%    Many times, we have a result but based on its number we can draw a conclusion or formulate a finding on top of it. 
%    Even if it was previous discussed in an earlier section, we need to re-state here. 
% \item \textbf{Contributions.} 
%    If we presented/discussed the main contributions of this research in the introduction, then we need to do again in the conclusion.
%    Do not repeat verbatim what was written in previous sections. 
%    In the conclusion, we expect contributions to be more detailed and linked to the results/findings when possible.
% \end{itemize}

% Avoid generic conclusion sentences that could be applied to anything. 
% For example, "Our technique showed good results which were beneficial to answer our research questions. 
% Our work can be used by other researchers to better understand our domain."
% Instead, go for more specific detailed results.
% For example, "Our technique showed a precision of 75\% which was 15\% higher than the baseline comparison. 
% Based on this we can see that ..."

% The final paragraph (or paragraphs) of the conclusion is about future research. 
% We can create a separate subsection for it if there are multiple paragraphs dedicated to future work. 
% Just be aware, it is not a good sign if future research content is longer than what we wrote for the previous paragraphs in the conclusion.


\newpage
\bibliographystyle{unsrt}
\bibliography{references}
\addcontentsline{toc}{section}{References}

\end{document}