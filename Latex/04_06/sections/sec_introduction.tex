% !TEX root = ../CedricDe Schepper2023_Thesis.tex

\section{Introduction}\label{sec:introduction}

% The introduction is probably the most important section of any academic work. 
% We start the introduction with a small contextualization on the thesis/paper subject. 
% This usually takes 1 to 3 paragraphs for a paper depending on the topic. 
% For a thesis, it is ok to write more paragraphs. 
% Please remember, these are just general suggestions.

% After the contextualization, we write one paragraph on the problem or motivation for this research. 
% We can complement with another paragraph to reinforce why the problem is important, or how it affects academia and/or industry. 

% Citations are very important in academic writing. 
% Try to put at least one citation (preferable more) per paragraph in the introduction's previous paragraphs. 
% Always use a citation when making a strong remark or statement to reinforce the point.
% Example of citation~\cite{demeyer2002}. For multiple citations put them all in the same cite command~\cite{vanbladel2020, parsai2020, njima2019, demeyer2002}. 
% Remember that citations are annotations, not parts of speech.
% Therefore do not use a citation as a substantive.

% After we successfully introduced the readers to the contextualization and problem/motivation, comes a paragraph clearly stating what is our research. 
% Usually, this paragraph begins with "In this paper/thesis, we ...".

% Now the reader understands the basics of our research and what we did to accomplish our goals. 
% The remaining paragraphs in the introduction can now describe a summary of the results, how previous research does not tackle what we did/accomplish, state the contributions for the research, or even an illustrative example of how the research improves the problem we described. 

% For Git-like repositories, try to put each sentence in a newline. 
% Since Git is line-based, it makes it easier the see changes between versions.

% The final paragraph of the introduction is an outline briefly describing the remaining sections. 
% Use the \textbackslash ref\{...\} command to reference Sections. 
% For example, in Section~\ref{sec:background}, we describe...

\comment{Still have to write this section}


% https://www.researchgate.net/publication/227419264_An_integrated_hybrid_approach_to_the_examination_timetabling_problem
% see introduction


% do something with this

% add rooms in introduction
The task of creating a university exam timetable can be reduced to a scheduling problem. The goal is to assign all
exams to available time slots in order to produce a schedule without conflicts. Additionally, the distribution of exams
has to be optimised as to provide a student with the highest chance of success. These requirements generally are defined
as hard and soft constraints. Hard constraints are requirements that have to be met in order to considered a feasible
solution, while soft constraints are preferred to be violated as little as possible. Not all timetabling problems might have
feasible solutions depending on the data set use
