% !TEX root = ../CedricDe Schepper2023_Thesis.tex

\section{Introduction}\label{sec:introduction}

% The introduction is probably the most important section of any academic work. 
% We start the introduction with a small contextualization on the thesis/paper subject. 
% This usually takes 1 to 3 paragraphs for a paper depending on the topic. 
% For a thesis, it is ok to write more paragraphs. 
% Please remember, these are just general suggestions.

% After the contextualization, we write one paragraph on the problem or motivation for this research. 
% We can complement with another paragraph to reinforce why the problem is important, or how it affects academia and/or industry. 

% Citations are very important in academic writing. 
% Try to put at least one citation (preferable more) per paragraph in the introduction's previous paragraphs. 
% Always use a citation when making a strong remark or statement to reinforce the point.
% Example of citation~\cite{demeyer2002}. For multiple citations put them all in the same cite command~\cite{vanbladel2020, parsai2020, njima2019, demeyer2002}. 
% Remember that citations are annotations, not parts of speech.
% Therefore do not use a citation as a substantive.

% After we successfully introduced the readers to the contextualization and problem/motivation, comes a paragraph clearly stating what is our research. 
% Usually, this paragraph begins with "In this paper/thesis, we ...".

% Now the reader understands the basics of our research and what we did to accomplish our goals. 
% The remaining paragraphs in the introduction can now describe a summary of the results, how previous research does not tackle what we did/accomplish, state the contributions for the research, or even an illustrative example of how the research improves the problem we described. 

% For Git-like repositories, try to put each sentence in a newline. 
% Since Git is line-based, it makes it easier the see changes between versions.

% The final paragraph of the introduction is an outline briefly describing the remaining sections. 
% Use the \textbackslash ref\{...\} command to reference Sections. 
% For example, in Section~\ref{sec:background}, we describe...



University exam timetabling is an unsolved problem encountered by every university's administration \cite{even1976}. Every year, significant manual effort has to be put into the creation of exam timetables. This need for automated tools capable of generating acceptable solutions has attracted the attention of researchers since 1963 \cite{gotlieb1963}. 

This task of creating exam timetables can be translated into a scheduling problem \cite{BurkeScheduling2004}. The goal is to assign all exams to available time slots and suitable rooms in order to produce a schedule without conflicts. Additionally, the distribution of exams has to be optimised as to provide a student with the highest chance of passing an exam. These requirements are generally defined as hard and soft constraints. Hard constraints are requirements that have to be met in order to be considered a feasible solution, while soft constraints are preferred to be violated as little as possible. Not all timetabling problems might have feasible solutions depending on the data set used.

The current relevance of this problem is showcased by the amount of algorithms developed, and the organisation of both timetabling conferences and competitions. The survey by Carter \cite{carter1986} details the research performed before 1986, while surveys such as the one by Siang Tan et al. \cite{joo2021} describe more recent implementations. However, no known research has been performed on the timetabling problem using the data and constraints of the University of Antwerp.

In this paper, we add a first contribution to the specific timetabling problem of the University of Antwerp. We do this by translating the exact use case into a formal scheduling problem. In order to solve the scheduling problem, we investigate which algorithm holds the most potential for this specific problem. This is done by reviewing the different types of algorithms available \cite{joo2021, kristiansenSurvey2013, chen2021, rong2009}. After proposing the implementation of a customised Tabu Search algorithm, we analyse the performance of this implementation. This is done by answering two questions based on quantitative and qualitative data. First, does Tabu Search succeed in creating a feasible solution without hard constraint violations? Second, can we optimise the solution to provide an acceptable exam distribution?

In Section \ref{sec:problem}, we look at the timetabling problem in detail, describing the constraints applied, its complexity, and the data required. Next, we look at the research already performed on this topic in Section \ref{sec:related-work}, including the benchmarks available and the different types of search algorithms applied. Section \ref{sec:method} focuses on the Tabu Search algorithm chosen and its implementation. Section \ref{sec:experiment} details the scoring metrics used to compare solutions and looks at fine tuning the needed hyper parameters. Finally, sections \ref{sec:results} and \ref{sec:conclusion} analyse the results obtained by executing several analyses, resulting in our final conclusions. 