% !TEX root = ../CedricDe Schepper2023_Thesis.tex

\section{Problem definition}\label{sec:problem}

 \subsection{Problem definition}

The task of creating a university exam timetable can be reduced to a scheduling problem. The goal is to assign all exams to available time slots in order to produce a schedule without conflicts. Additionally, the distribution of exams has to be optimised as to provide a student with the highest chance of success. These requirements generally are defined as hard and soft constraint. Hard constraints are requirements that have to be met in order to considered a feasible solution, while soft constraints are preferred to be violated as little as possible. Not all timetabling problems might have feasible solutions depending on the data set used. After consulting with the responsible parties for the scheduling of the exams for the FTI and FWET faculties, the following constraints were determined:
\begin{itemize}
    \item Constraint 1: Each exam is scheduled to a timeslot.
    \item Constraint 2: Each exam is scheduled to a room of the correct type.
    \item Constraint 3: The number of students enrolled in an exam cannot exceed the capacity of the scheduled room.
    \item Constraint 4: No student must be scheduled to more than 1 exam on the same day.
    \item Constraint 5: The number of days between exams must be maximised
\end{itemize}
Constraint 1 to 4 are considered to be hard constraints with contraint 5 being the sole soft constraint. Formally this means that the first 4 constraints must be satisfied in order to provide a feasible solution, with constraint 5 being taken into account to improve it.

In order to solve this scheduling problem, some data must be provided or generated:

\begin{description}
   \item [Students] The set of all students that are enrolled in exams and will impact the timetabling.
   \item[Exams] The set of all exams that must be scheduled. Every exam has its own set of students that are enrolled which will allow to verify if two exams have common students and would thus conflict if they occur on the same day. Additionally, an exam belongs to a faculty and is of a certain type. Two examples of exam types are oral and written exams.
   \item[Rooms] The set of rooms present for examinations. Each room has a room type, a student capacity and  list of faculties it is available for. An exam can only be scheduled in a room if the capacity is sufficient, the type of exam is possible within the room, and the room is available to the faculty to which the exam belongs.
    \item[Time slots] The set of all time slots that can be used to schedule exams. A time slot is determined by its date, exam time, and room. The exam time determines the amount of times a room can be used per day.
    \item[Periods] Set of periods, one for each date within the examination schedule. Each period contains all time slots of that date. By checking all students that are enrolled within exams scheduled to a time slot of a period, the amount of constraint 4 ("no student must be scheduled to more than 1 exam on the same day") violations can be determined. These periods can be generated by adding a time slot for every room and time combination. The amount of time slots in a period corresponds to $|\text{Rooms}| * |\text{Exam Times}|$.
\end{description}

